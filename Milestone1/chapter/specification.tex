\chapter{Specification}\label{ch:specification}
    \section{Description}

        This project includes the development of a wearable device that is able to detect when the user is/has moved,
        and a remote device attached to their walking aid, that is able to remind them to use it if they have not done
        so.\ A prerecorded message from a family member or relative will act as the notification.\ Both devices have to
        be discrete and unobtrusive, as the users are known to be easily startled by new and foreign devices near them,
        as well as with lights/vibration and other stimulus easily confusing them.\ The message will be stored locally,
        either on the hardware products inbuilt storage, or an external SD card.\ In the eventuality that the user is
        deaf or hard of hearing, the device will also have to have support for alternatives stimuli, such as vibration
        or light based feedback.

    \section{Software Behaviour}

        The wearable device must communicate with the walking aid device in order for the project to be a success. This
        communication, can happen. As soon as movement is detected, the transmitter will
        send message to the wearable device, letting it know that the user started walking. In this way the wearable
        device will be able to tell when the patient has used the walking aid and is actually walking. Otherwise, as
        mentioned before a message will be played to remind the patient to take their walking aid with them. The device
        will cease warning the user once the system detects that the user has used the walking aid and begun walking.

        Furthermore, dementia all types of people at varying ages. That means that the wearable device must be able to
        fit on anyone's wrist. Because we do not want patients to feel uncomfortable while using the wearable device, it
        will be designed to look like a watch. That means that the user, who in this case is a dementia patient, will
        not interact with the watch, by any means of having to select an option on the device or even to change a
        setting. Apart from the device's appearance and dimensions, the device should be power efficient, as the
        patients most of the time will forget to charge it. Aside the device's appearance and dimensions, the wearable device's material must be durable in the case it is dropped by the user, that is why the material of the wearable device will be ABS plastic, which is an opaque thermoplastic and amorphous polymer.

        The device will also include a fall detection functionality. If the development time allows, we will integrate
        this capability into the device. A tri axial accelerometer will be used to determine the user's hand height, and
        more particularly the device's height. If the user's hand is extremely close to the ground, it is likely that
        the user has fallen. If the device detects that the user has been on the ground for an extended amount of time,
        it will begin informing a relative or a nearby caregiver.

        In terms of the security of the system, the devices, wearble device and the walking aid, will communicate only with each other in pairs. This can be achieved by using a unique synchronization mechanism between each device, so that the network will know that a single wearble device belongs to a single user, as it is belongs to a single walking aid. As a result, an external device will be unable to commuicate with the walking aid or even the wearble device within the network.
