\chapter{Project schedule}\label{ch:schedule}
    \section{Gantt Chart}
    We built a Gantt Chart for the project schedule to allow us to keep track of the time. We couldn't create tasks that
    would indicate a slippage since it would disrupt the critical path. As a result, we increased the time alloted to
    each assignment, reflecting the slippage. We included slippage due of other university homework, exams and even
    illness.

    As shown in Figure \ref{fig:gantt_chart}, the highlighted tasks illustrate the project's critical path. We devided
    the project into three Sprints because we are utilizing the scrum methodlogy. The first Sprint consists only of
    require collection and project planning. The second Sprint includes the sprint plan, as well as the software and
    hadrware designs and implementations. The third Sprint contains the sprint plan as well as the system's final
    implementation, in which the walking aid will connect with our system, the wrist device. The final task of each
    Sprint is a milestone, signaling the Sprint's completion.

    \begin{figure}[H]
      \includegraphics[width=\linewidth]{graphics/Gantt_Chart.png}
      \caption{Gantt Chart}
      \label{fig:gantt_chart}
    \end{figure}

    \section{Activity Network Diagram}

    In terms of scheduling, an activity diagram is just a significant as designing a Gantt Chart. We merged several of
    the jobs that were introduced to Gantt Chart into one task to make the activity network diagram more undestandable.
    However, this has no bearing on the total time requried to finish the project. According to Figure
    \ref{fig:activity_network_diagram}, the total time requried is 107 working days.

    \begin{figure}[H]
      \includegraphics[width=\linewidth]{graphics/AND.drawio.png}
      \caption{Activity Network Diagram}
      \label{fig:activity_network_diagram}
    \end{figure}
