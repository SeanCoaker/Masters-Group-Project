\chapter{Project Management}\label{ch:schedule}

    \section{Team Roles}
        Within this section, we will detail roles needed to be fulfilled within our team that aligns with the context of our project and our chosen software lifecycle methodology. Each role will be assigned team members that are responsible for the tasks being undertaken by said role. Due to the nature of our scrum methodology, we will be avoiding the use of a designated team leader and instead opting for a self-managing team \cite{parafianowicz_2019}. A list of our team roles can be seen below, along with their assignees and detail about what each role entails.

        \subsection{Scrum Master}
            \textbf{Assignee: } Sean Coaker\newline
            \textbf{Description: } The scrum master within our project will not act as a team leader, but will instead facilitate and manage the process of the scrum software lifecycle methodology \cite{bass_2014}. Due to the adjustments we have made to our scrum methodology, such as avoiding daily meetings due to university commitments, our scrum master will need to ensure that our team adheres to our adjusted methodology and time plan set out within this document. Therefore, our scrum master will need a clear understanding of the team's selected software lifecycle methodology and should regularly reference it to provide consistent guidance to the team when developing the project.

        \subsection{Full-Stack Developer}
            \textbf{Assignee: } All Team Members (Sean Coaker, Pedro Caetano, Matthew Culley, Panayiotis Melios)\newline
            \textbf{Description: } Our full-stack developers will handle the code development for all aspects of the project. They will develop code for both the walking aid and wearable devices, implementing code that allows our project to meet the functional and non-functional requirements detailed within this document. The developers will need to quickly adjust to the new software and hardware they encounter during the development of this project to ensure that the final product is developed on time. As mentioned previously, our team is self-managed and therefore regular communication will need to occur between developers to keep everyone up to date with progress being made.

        \subsection{Communications Officer}
            \textbf{Assignee: } Sean Coaker\newline
            \textbf{Description: } To provide communication between the development team and the client, we elected a communications officer. The communications officer shall be the main point of contact with the client, dealing with queries, organising meetings and ensuring that the client requirements are communicated correctly to the development team. The communications officer has so far organised initial meetings with the client and submitted documented user requirements to the client, which they confirmed they are content with.

        \subsection{Prototype Design Officer}
            \textbf{Assignee: } Pedro Caetano\newline
            \textbf{Description: } Due to the nature of our project, we needed to assign a prototype design officer to handle the design and development process of the device casing. We elected Pedro for this due to his previous experience in working with 3D printers and feel that his experience can transfer well into developing the designs for our device casing, along with being able to transfer these designs into 3D printed useable prototypes.

        \subsection{Testers including Quality Assurance responsibilities}
            \textbf{Assignee: } All Team Members (Sean Coaker, Pedro Caetano, Matthew Culley, Panayiotis Melios)\newline
            \textbf{Description: } As mentioned previously, our team is self-managed meaning we wanted to avoid the use of leadership roles such as team leaders, testing leaders and quality assurance leaders. For this role, all team members will be expected to perform testing procedures that align with our testing strategy. Along with this, each team member will also take on quality assurance responsibilities to ensure that our product is of the highest quality the team can produce. Our testing role also includes the responsibilities of documenting test results to allow the team to collate them at the development of our milestone 3 document for proof of the success of our product.

        \subsection{Documentation Developers}
            \textbf{Assignee: } All Team Members (Sean Coaker, Pedro Caetano, Matthew Culley, Panayiotis Melios)\newline
            \textbf{Description: } As we are a small team, it is a requirement that all team members contribute to the development of the documentation required to be completed within this project. The fact that all members assigned to the development of the documentation are also full-stack developers can be beneficial here as they can provide an in-depth understanding of the developed system when detailing it within our documentation.

    \section{Time Plan}

        \subsection{Gantt Chart}
            We built a Gantt Chart for the project schedule to allow us to keep track of the time. We couldn't create tasks that
            would indicate a slippage since it would disrupt the critical path. As a result, we increased the time allotted to
            each assignment, reflecting the slippage. We included slippage due to other university homework, exams and even
            illness.

            As shown in Figure \ref{fig:gantt_chart}, the highlighted tasks illustrate the project's critical path. We divided
            the project into three Sprints because we are utilizing the scrum methodology. The first Sprint consists only of
            required collection and project planning. The second Sprint includes the sprint plan, as well as the software and
            hardware designs and implementations. The third Sprint contains the sprint plan as well as the system's final
            implementation, in which the walking aid will connect with our system, the wrist device. The final task of each
            Sprint is a milestone, signalling the Sprint's completion.

            \begin{figure}[H]
              \includegraphics[width=\linewidth]{graphics/Gantt_Chart.png}
              \caption{Gantt Chart}
              \label{fig:gantt_chart}
            \end{figure}

        \subsection{Activity Network Diagram}

            In terms of scheduling, an activity diagram is just a significant as designing a Gantt Chart. We merged several of
            the jobs that were introduced to the Gantt Chart into one task to make the activity network diagram more understandable.
            However, this has no bearing on the total time required to finish the project. According to Figure
            \ref{fig:activity_network_diagram}, the total time required is 107 working days.

            \begin{figure}[H]
              \includegraphics[width=\linewidth]{graphics/AND.drawio.png}
              \caption{Activity Network Diagram}
              \label{fig:activity_network_diagram}
            \end{figure}
