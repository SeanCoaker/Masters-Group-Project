\chapter{Specification}
\label{chap:specification}

In this chapter, we'll go through the project in further depth.
More specifically, we will outline a proposed solution as well as describing the
product.

\section{Description}

The challenge of this project is to develop an effective system that interacts
with users, who are dementia patients. This project will support the development
of a wearable device including an application. The dementia patient or user will
be encouraged to use the walking assistance in this application. A recorded
message from a family member or relative will act as the notification. The device
will detect if the user is sleeping and if is not, the device will only then be
activated, since we do not want the device to begin notifying the user while he
is sleeping, as the user might be startled by the notice.

The user does not need to log in or register with the system since the device
will obtain the necessary data through an SD card, such as the familiar voice of
a relative or family member. Although dementia patients may be deaf, the
wearable device will support a solution for deaf individuals, such as reminding
them to take their walking aid with them without utilising any audio, such as a
slight vibration.

\section{Software behaviors}

The wearable device must communicate with the walking aid device in order for
all of this to be happen. This communication, can be happened by using a
transmitter. As soon as movement is detected, the transmitter will send message
to the wearable device, letting it know that the user started walking. In this
way the wearable device will be able to tell when the patient has used the
walking aid and is actually walking. Otherwise, as mentioned before a message will
be played to remind the patient to take their walking aid with them. The device
will cease warning the user once the system detects that the user has used the
walking aid and begun walking.

Furthermore, dementia all types of people at varying ages. That means that the
wearable device must be able to fit on anyone's wrist. Because we do not want
patients to feel uncomfortable while using the wearable device, it will be
designed to look like a watch. That means that the user, who in this case is a
dementia patient, will not interact with the watch, by any means of having to
select an option on the device or even to change a setting. Apart from the
device's appearance and dimensions, the device should be power efficient, as the
patients most of the time will forget to charge it.

The device will also include a fall detection functionality. If the development
time allows, we will integrate this capability into the device. A tri axial
accelerometer will be used to determine the user's hand height, and more
particularly the device's height. If the user's hand is extremely close to the
ground, it is likely that the user has fallen. If the device detects that the
user has been on the ground for an extended amount of time, it will begin
informing a relative or a nearby caregiver.
