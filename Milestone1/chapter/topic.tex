\chapter{Topic} \label{ch:topic}

        Within this chapter we will detail the problem our project aims to solve along with the limitations that could
        effect our project and an analysis of the current solutions our project would have to compete with. We will
        provide some background research on dementia patients and their issues with forgetting their walking aids when
        moving and how wearable devices can have a psychological impact on them. Finally, we will detail our current
        progress consisting of two initial meetings with our client, and our takeaways from these meetings.

    \section{Background}

        Within this chapter we will detail the problem our project aims to solve along with the limitations that could
        affect our project and an analysis of the current solutions our project would have to compete with. We will
        provide some background research on dementia patients and their issues with forgetting their walking aids when
        moving and how wearable devices can have a psychological impact on them. Finally, we will detail our current
        progress consisting of two initial meetings with our client, and our takeaways from these meetings.

        The client came to us with the idea of developing a product for those suffering with dementia, a syndrome that
        is usually associated with a declining functionality of the brain. Dementia can manifest itself with a range of
        symptoms, more well-known ones including memory loss, loss of mental sharpness, or loss of the use of
        language.~\cite{nhs_choices} More importantly to our project though, one other symptom can be a loss in movement
        skills, or an increased difficulty in moving. Dementia has been recognised something that causes an increase in
        fall risk within patients
        source~\cite{doorn_gruber-baldini_zimmerman_hebel_port_baumgarten_quinn_taler_may_magaziner_et_al._2003}. As
        dementia affects more elderly persons, falls are often more dangerous to them. Because of this, it is not
        uncommon for dementia patients to be using walking aides to help mitigate this. Obviously though, a walking aid
        is only effective if it is used by the patient. Something that was made clear to us with our meeting with the
        client.

    \section{The Problem}
        Upon completion of our initial meeting with the client, we clarified the motivation behind this project and the
        problem that we are working together to solve. That problem is to develop a solution that detects when a
        dementia patient is moving without their walking aid and reminds the patient (with a recorded message by a
        friend or a relative) to take their walking aid with them. Initial discussions between ourselves and the client
        identified current issues with dementia patients feeling uncomfortable in being forced to wear foreign objects,
        meaning we would need to take this into account when developing our solution. We also clarified that dementia
        patients get easily alarmed and frightened by generic “obnoxious” alarms, often associating them with those
        notifying them of danger, such as fire alarms. The client suggested that we facilitate a recording feature
        within our solution that would allow recognisable voices to the dementia patient to remind them to use their
        walking aid.

        \subsection{Similar Solutions}
            Current solutions include the use of locator systems that allow dementia patients to easily track down
            valuables such as keys or a wallet. Such systems include the Tile ecosystem which allows a dementia patient
            to attach a Tile device to their valuables and then use a smartphone app to fire an alarm from the Tile
            device that notifies the patient of the location of their valuable. As previously mentioned, dementia
            patients can get frightened and disorientated by the sounds of alarms often associating them with danger
            rendering these forms of solutions unsuitable for our problem. This is without considering how difficult a
            dementia patient mind finds navigating through a smartphone device to open an application and request their
            Tile device to ring an alarm to help them identify the location of their valuable. Another problem with such
            an approach though, is the lack of functionality for sending an alarm to the patient if they start walking
            without the aide. Solutions such as Tile would help a patient find the aide if it had been misplaced but
            implementing functionality to remind the patient to find the aide if they have already started walking
            without it, is sadly limited. Other more old-fashioned systems that carers may use to notify themselves that
            their patient is moving include hanging items from door frames that clatter together when the patient walks
            through the door or adding pressure pads under door mats that sound an alarm when the patient steps on the
            door mat. But what we are trying to expand upon with our solution is the protection of dementia patients
            that are alone and wanting to move around their home or ward. Meaning that door mat pressure pad solutions
            and methods for alarming a carer would be insufficient. We also would like our design to be more elegant,
            with those rudimentary solutions, if the patient is walking fine, with the aide, they will still get an
            alarm from the pressure pad. Or they will still have to remove the hangars from the aide. A well implemented
            system could alleviate both those problems, and increase the quality of life of the patient, even if only by
            a little.

        \subsection{Limitations}
            Our main limitation for our project is that we need to develop a discrete device that will not make the
            dementia patient feel uncomfortable in any way or will minimise discomfort to an extent where it is
            acceptable for day-to-day life. Early plans for the device lean towards a watch style device that the
            patients can wear on their wrist to track their movement. But if we are to create a wrist wearable device
            then we would need to ensure that the footprint of device is small enough to be worn on a wrist. This limits
            the hardware that we can feasibly use for our project. We also need to consider the number of devices being
            used and how they can all be fit within a wrist device. The head of Swansea University’s Embedded Systems
            module has kindly offered to supply us with ESP32 based TinyPICO devices which would be suitable for this
            project due to its small form factor. We would need to consider how an accelerometer could be attached to
            the TinyPICO to allow both devices to fit within a watch casing. Even still, if we could find hardware small
            enough to build into a watch-based prototype, we may still run into issues with the patients wanting to use
            the product. During our meeting we were told that patients already had to wear wrist tags or similar items.
            Adding more items to the list of items the patient needs to wear is unlikely to be well received. Our form
            factor perhaps could take the form of something that clips onto something the patient is already wearing,
            such as clothes or belts, or a tag that they already use. Other limitations for the wrist device part of the
            system are that it should not contain any LEDs, vibration motors or alarms to avoid startling the dementia
            patient. Avoiding the use of LEDs would be of benefit to us here as it would allow the watch device to save
            battery during operation. On the topic of saving battery, another limitation to the watch device is that it
            would need to be power efficient enough to avoid the dementia patient needing to frequently charge the
            device. The patient needing to do this would work against our goal of creating a user-friendly experience
            for them. The ESP32 chips included on the TinyPICO boards utilise a system called 'deep sleep', which
            effectively powers down certain modules connected to the board. We could theoretically create a system here
            that fires an interrupt when the accelerometer detects movement, then forcing the ESP32 to wake up and
            handle the interrupt. Thus, meaning that when the patient is static, the device can be in a 'deep sleep'
            state to save battery. Limitations for the device being fitted to the walking aid are far less. With this
            device, we do not need to consider a small form factor as it is not being worn but will be using a TinyPICO
            for this device too for consistency. Our limitation with this device is to also disable the use of LEDs to
            avoid startling the patients, and to include a speaker and microphone to allow a relative or carer to record
            a voice note which will be played to remind the patient to take their walking aid with them when moving. The
            TinyPICO boards include a very minimal amount of storage space and so we may need to include a SD card to
            store the voice notes on. We will also be limited to the budget of £150 that we have been assigned and must
            ensure that all the devices needed to build the system can be purchased within our budget.

    \section{Current Work}
        As stated earlier in this chapter, we have held initial meetings with the client where we have clarified the
        problem they aim to solve with this project and outlining the project scope. On the 18th of Novemeber we held an
        introductory meeting with the client where we gained an understanding of what the problem is and what kind of
        system the client was expecting to be produced. We clarified that we would need to gain our supply of hardware
        ourselves and that a budget of £150 would be allocated to us to aid with the procurement of the necessary
        hardware devices. However, within this initial meeting we failed to identify the final direction the client
        wanted the project to head down and instead came away with the option of either developing a wearable device
        that would detect when the dementia patient was moving, or to use a non-wearable device such as a pressure
        blanket that would detect when the patient had got up from where they were static. We agreed with the client
        that we would schedule a second meeting for the 25th of Novemeber and within that time analyse the advantages
        and disadvantages to each method of developing the solution. We then agreed that we would return with a solution
        that we thought would best suit the design brief and that would best suit the development talent available to us
        within our team.

        Within our own intra-team meeting we decided upon building and developing a wearable device solution over a
        non-wearbale solution due to the extra features that could be included into a wearable device such as a fall
        detection system, a system recommended to be included by the client. We felt that despite a non-wearable device
        being a plausible route to take the project down, that factors such as a dementia patient moving off a pressure
        pad without actually standing up and walking would diminish the effectiveness of our solution.

        On the 25th of November we hosted our second meeting with the client and established the team's preferred route
        for the development of this project. The client was content with this and agreed that the solution should be
        developed as a wearable device. We finalised the £150 budget with the client and agreed that our next steps
        would be to complete our milestone 1 document, including user requirements, and compiling a list of necessary
        hardware to develop the project. Our next meeting with the client is scheduled for December 16th where we will
        finalise the user requirements and compile a list of hardware to be purchased with the budget made available to
        us.

    \section{Project Aims}
        In order for this project to be successfully, there a few broad targets and ambitions we are looking to fulfil
        with our final system. First of all, the system needs to be able to effectively remind its user that they have
        forgotten their walking aid, and actively encourage them to use it. The goal is to prevent accidental falls,
        which according to our client, are a common problem with dementia patients or senior citizens, who often forget
        or don't want to use their walking aid. This must also easily integrate with the patients existing lifestyle, as
        often they have a large amount of inertia and would prefer to avoid change. This is why we are purseing a small
        wearable device, like similar solutions in the market they may have experienced before, and incorporating a
        familiar voice to help calm and ease them into using our system.
