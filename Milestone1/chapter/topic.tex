\chapter{Topic} \label{ch:topic}

    This chapter details the problem and tasks our project aims to solve along with the limitations that could affect
    our project, including an analysis of the current solutions our project has to compete with.\ We will provide
    background research on dementia patients and senior citizens with similar conditions (hereby depicted as the
    'User(s)') and their issues with forgetting their walking aids, and how wearable devices can have a varying
    psychological impact on them.\ Finally, we will detail our current progress consisting of two initial meetings
    with our client, and our takeaways from these engagements.

    \section{Background}

        The client came to us with the idea of developing a product for those suffering from dementia, a syndrome that
        is usually associated with a declining functionality of the brain.\ Dementia can manifest itself with a large
        range of symptoms, most commonly including memory loss, loss of mental sharpness, or loss of the use of
        language~\cite{nhs_choices}.\ More importantly to our project, one other symptom can be a loss in movement
        skills or an increased difficulty in moving.\ Dementia has been recognised as an illness that causes an
        increase in falls within
        patients~\cite{doorn_gruber-baldini_zimmerman_hebel_port_baumgarten_quinn_taler_may_magaziner_et_al._2003}.\ As
        dementia affects mainly more elderly population, falls are often more dangerous to them.\ It is therefore not
        uncommon for dementia patients to be using walking aids to help mitigate this, a walking aid is only effective
        if it is used by the patient, and as explained by the client, this is something the users may struggle with
        remembering to use.

    \section{The Problem}
        Upon completion of our initial meeting with the client, we clarified the motivation behind this project and the
        problem that we are working together to solve.\ The problem is to develop a solution that detects when a
        user is moving without their walking aid and reminds the patient (with a recorded message by a
        friend or a relative) to take their walking aid with them.\ Initial discussions between ourselves and the client
        identified current issues with users feeling uncomfortable in being forced to wear foreign objects,
        meaning we would need to take this into account when developing our solution.\ We also clarified that users get easily alarmed and frightened by generic “obnoxious” alarms, often associating them with those
        notifying them of danger, such as fire alarms.\ The client suggested that we facilitate a recording feature
        within our solution that would allow recognisable voices to the user to remind them to use their
        walking aid.

        \subsection{Similar Solutions}
            Current solutions include the use of locator systems that allow users to easily track down valuables such as
            keys or a wallet.\ Such systems include the Tile ecosystem which allows a user to attach a Tile device to
            their valuables and then use a smartphone app to trigger an alarm from the Tile device that notifies the
            patient of the location of their valuable.\ As previously mentioned, users can get frightened and
            disorientated by the sounds of alarms often associating them with danger rendering these forms of solutions
            unsuitable for our problem.\ This is without considering how difficult a user may find navigating through
            a smartphone device to open an application and request their Tile device to ring an alarm to help them
            identify the location of their valuable.\ Another problem with such approaches is the lack of functionality
            for sending an alarm to the patient if they start walking without the aid.\ Solutions such as Tile would
            help a patient find the aid if it had been misplaced by implementing functionality to remind the patient
            to find the aid if they have already started walking without it, is lacking entirely.\ Other more legacy
            systems that carers may use to notify themselves that their patient is moving include hanging items from
            door frames that clatter together when the patient walks through the door or adding pressure pads under doormats that sound an alarm when the patient steps on the doormat.\ Our solution is focussing on the protection
            of users that are alone and wanting to move around their home or ward, meaning that doormat pressure pad-based solutions and methods for alarming a carer would be insufficient.\ We also would like our design to be
            more elegant than those rudimentary solutions, if the patient is walking with the aid, they will still get
            an alarm from the pressure pad.\ A well-implemented system could alleviate both those problems, and increase
            the quality of life of the patient and facilitate the carer.

        \subsection{Limitations}
            Our main limitation for our project is that we need to develop a discrete device that will not make the
            users feel uncomfortable in any way and will minimise discomfort to an extent where it is
            acceptable for day-to-day life.\ Early plans for the device lean towards a watch-style device that the
            patients can wear on their wrist to track their movement.\ If we were to create a wrist wearable device
            then we would need to ensure that the footprint of the device is small enough to be worn comfortably.\ This limits
            the hardware that we can feasibly use for our project.\ We also need to consider the hardware being
            used and how they can all be fit within a wrist device.\ The head of Swansea University’s Embedded Systems
            module has kindly offered to supply us with ESP32 based TinyPICO devices which would be suitable for this
            project due to their small form factor.\ We would need to consider how an accelerometer could be attached to
            the TinyPICO to allow both devices to fit within a watch casing.\ If we could find hardware small
            enough to build into a watch-based prototype, we may still run into issues with the patients wanting to use
            the product.\ During our meeting we were told that patients already had to wear wrist tags or similar items.
            Adding more items the patient needs to wear is unlikely to be well received.\ Our form factor could take the
            shape of something that clips onto what the patient is already wearing, such as clothes or belts, or a tag
            that they already use.

            Other limitations for the wrist device are that it should not contain any strong LEDs, obnoxious vibration
            motors or alarms to avoid startling the user.\ Avoiding the use of LEDs would be of benefit to us here as it
            would allow the watch device to save battery during operation.\ Another limitation to the watch device is
            that it would need to be power efficient and avoid the users needing to frequently charge the device.\ The
            patient needing to do this would work against our goal of creating a user-friendly experience for them.\ The
            ESP32 chips included on the TinyPICO boards utilise a facility called 'deep sleep' or hibernation states,
            which effectively powers down certain modules connected to the board.\ We could create a system here that
            fires an interrupt when the accelerometer detects movement, then forces the ESP32 to wake up and handle the
            interrupt.\ The device could therefore be in sleep whilst the patient is static, to save battery.

            The device attached to the walking aid has much fewer limitations, and so we do not need to consider such a
            small form factor as it is not being worn but will still be using a TinyPICO to encourage compatibility and
            interchangeable/modular development experiences.\ Our limitation with this device is to also disable the use
            of LEDs to avoid startling the patients and to include a speaker and microphone to allow a relative or
            carer to record a voice note which will be played to remind the patient to take their walking aid with them
            when moving.\ The TinyPICO boards include a very minimal amount of storage space and so we may need to
            include an SD card to store the recordings.\ We will also be limited to the budget of £150 that we have been
            assigned and must ensure that all the devices needed to build the system can be purchased within our budget.

    \section{Current Work}
        As stated earlier in this chapter, we have held initial meetings with the client where we have clarified the
        problem they aim to solve with this project and outlined the project scope.\ On the 18th of November, we held an
        introductory meeting with the client where we gained an understanding of what the problem is and what kind of
        system the client was expecting to be produced.\ We clarified that we would need to gain our supply of hardware
        ourselves and that a budget of £150 would be allocated to us to aid with the procurement of the necessary
        hardware devices.\ However, within this initial meeting, we failed to identify the final direction the client
        wanted the project to head down and instead came away with the option of either developing a wearable device
        that would detect when the users were moving, or to use a non-wearable device such as a pressure
        blanket that would detect when the patient had got up from where they were static.\ We agreed with the client
        that we would schedule a second meeting for the 25th of November and within that time analyse the advantages
        and disadvantages to each method of developing the solution.\ We then agreed that we would return with a solution
        that we thought would best suit the design brief and that would best suit the development talent available to us
        within our team.

        Within our intra-team meeting we decided upon building and developing a wearable device solution over a
        non-wearable solution due to the extra features that could be included in a wearable device such as a fall
        detection system, a system recommended to be included by the client.\ We felt that despite a non-wearable device
        being a plausible route to take the project down, that factors such as users moving off a pressure
        pad without actually standing up and walking would diminish the effectiveness of our solution.

        On the 25th of November, we hosted our second meeting with the client and established the team's preferred route
        for the development of this project.\ The client was content with this and agreed that the solution should be
        developed as a wearable device.\ We finalised the £150 budget with the client and agreed that our next steps
        would be to complete our milestone 1 document, including user requirements, and compile a list of the necessary
        hardware to develop the project.\ Our next meeting with the client is scheduled for December 16th where we will
        finalise the user requirements and compile a list of hardware to be purchased with the budget made available to
        us.

    \section{Project Aims}
        For evaluation purposes after this project, we will detail within this section a list of aims
        that should be met along with lower-level objectives that define a set of criteria that will allow us to meet
        said aims.\ An evaluation within milestone 3 will look to compare the final product produced against the aims set
        out in this section in an attempt to gauge how successful our project is.\ A list of our aims and their lower
        level objectives can be seen below.

        \begin{itemize}
            \item Develop a solution that reminds the user(s) to take their walking aid with them when attempting to walk.
            \begin{itemize}
                \item We should create a wearable device that includes a tri-axial accelerometer to detect when the
                    user(s) have started moving.
                \item A device should be connected to the walking aid that also contains a tri-axial accelerometer that
                    will be used to detect if the walking aid is moving whilst the wearable device is moving.
                \item The walking aid device should contain a microphone and speaker that allows a carer to record a
                    reminder for the user(s) in an attempt to avoid startling the user(s), which could happen when using
                    generic alarms.
            \end{itemize}
            \item Create prototype devices that avoid startling the user(s) and avoid making the user uncomfortable.
            \begin{itemize}
                \item The wearable device should be designed and built such that it is inconspicuous in an attempt
                    to not draw attention to it from the user(s).
                \item Unless the user(s) is deaf, we should avoid the use of LEDs, vibration motors, and the use
                    of generic alarms.
                \item The wearable device should be developed to be worn on the wrist rather than on a more
                    uncomfortable body part such as the neck.
            \end{itemize}
            \item Produce a solution that the client concludes is satisfactory.
            \begin{itemize}
                \item Produce a document of user requirements and receive confirmation from the
                    client that the requirements are sufficient.
                \item Allow for changes in requirements during project
                    development.\ Our chosen agile methodology will allow for easy integration of changed user
                    requirements here.
                \item Develop the solution such that it is compliant with the user requirements.
                    \end{itemize}
        \end{itemize}
