\chapter{Topic}
\label{chap:topic}

Within this chapter we will detail the problem our project aims to solve along with the limitations that could effect our project and an analysis of the current solutions our project would have to compete with. We will provide some background research on dementia patients and their issues with forgetting their walking aids when moving and how wearable devices can have a psychological impact on them. Finally, we will detail our current progress consisting of two initial meetings with our client, and our takeaways from these meetings.

\section{Background}

\section{The Problem}
    Upon completion of our initial meeting with the client, we clarified the motivation behind this project and the problem that we are working together to solve. That problem is to develop a solution that detects when a dementia patient is moving without their walking aid, and reminds the patient (with a recorded message by a friend or a relative) to take their walking aid with them. Initial discussions between ourselves and the client identified current issues with dementia patients feeling uncomfortable in being forced to wear foreign objects, meaning we would need to take this into account when developing our solution. We also clarified that dementia patients get easily alarmed and frightened by generic alarms, often associating them with danger notifying alarms such as fire alarms, with the client suggesting that we facilitate a recording feature within our solution that would allow recognisable voices to the dementia patient to remind them to use their walking aid.

\section{Current Work}
    As stated earlier in this chapter, we have held initial meetings with the client where we have clarified the problem they aim to solve with this project and outlining the project scope. On the 18th of Novemeber we held an introductory meeting with the client where we gained an understanding of what the problem is and what kind of system the client was expecting to be produced. We clarified that we would need to gain our supply of hardware ourselves and that a budget of £150 would be allocated to us to aid with the procurement of the necessary hardware devices. However, within this initial meeting we failed to identify the final direction the client wanted the project to head down and instead came away with the option of either developing a wearable device that would detect when the dementia patient was moving, or to use a non-wearable device such as a pressure blanket that would detect when the patient had got up from where they were static.
