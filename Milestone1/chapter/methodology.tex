\chapter{Methodology}
\label{ch:methodology}

Due to the nature of our team and project. We thought it best to implement an agile based development methodology. As a small team, it is easy for us to communicate with each other about the project, and easier for us to be aware of what the other members of our team are doing, key factors for choosing an agile methodology.
The question then is, which agile based methodology to use? There are a variety of methodologies our there each with their own pros and cons, which we will address some of which shortly. But based on these we decided to use a scrum-based development methodology. Scrum is a combination of iterative and incremental development. Allowing for a best of both worlds approach of having early builds working, but also being agile and able to add requirements during the development process. \cite{srivastava_2017_scrum}

The scrum methodology is rather simple. To effectively work, it requires collaboration between an appointed ‘Scum Master’ and the rest of the team, as well as a ‘Product Owner’. These members will work in close collaboration other multiple, continuous iterations of the software build to create a finished product. The role of the scrum master is to eliminate impediments \cite{srivastava_2017_scrum}, or in other terms, to create the conditions that allow the development team to work in their most effective manner, they take a leading role in choosing what sprints the team work on.

A typical scrum workflow consists of iterative scrum cycles, typically lasting one to three weeks each. Initially a requirements backlog is created.  This is done in collaboration with the client and is the documentation that describes all the requirements that the software must meet. It is essentially a description of the product that the team are aiming to build. Once an initial backlog is created, the scrum cycles can start. As before mentioned, each scrum cycle lasts typically up to three weeks. The cycle kicks of with the team, headed by the scrum master deciding which requirements to prioritise and implement in the upcoming cycle. The aim is to have a potentially shippable product at the end of each sprint, although this may not be possible immediately. Once the team have allocated the suitable requirements to the sprint, the sprint begins, the team works on the product, meeting each day in what are known as ‘scrums’ to review progress made on each day. At the end of the sprint, the team should have made a shippable product. The team will also meet and hold a sprint review, along with the product owner. That demonstrates the product and how it has been developed since the last sprint. The team will also review the previous sprint, to see if any changes need to be made in anticipation of the next one.\cite{nuevo_2011_scrumbased}

In our case, we will be slightly modifying the implementation of scrum. The above-mentioned workflow is designed as a framework for full time software developers, which we are not. As much as we may perhaps wish too, we will not be able to meet daily to discuss the project. As students it is an unrealistic expectation that we will be able to hold daily scrums, but we can still follow the methodology. We will aim to hold scrums at frequent intervals, in periods perhaps of every week, to keep tabs on the current sprints running. We can also compensate for this by being realistic in what we can achieve, and what we set as the sprint’s goals. 

Ultimately, we chose to use scrum for a variety of reasons. Primarily of all, we realised that as a small team, frequent communication would be easy to achieve, so we would be able to make use of this framework in the most effective manner, compared to other methodologies. We would be able to meet frequently and use that to ensure progress is made on the project. Scrum works well with this. Using scrum would enable us to work together to frequently get working products out and allow us to put ourselves in a position to implement more features. We also feel that using scrum would allow us to have a higher quality product, as the frequent iterations during sprints would allow us to streamline our implementation and constantly improve it. It also allows us to minimise risk. Take the waterfall methodology for example, where implementation begins a lot later in the lifecycle. Using scrum, where we start programming earlier, gives us more time to deal with any difficulties we encounter. It allows us to spot problems earlier, have longer to address them, or to have time to plan alternative implementations that avoid such problems. 

We chose scrum over other agile methodologies for a few reasons. But mainly because we thought it was the best fit for our team and situation. Other methodologies, like extreme programming, could have worked well for us, but didn’t suit our situation as students as well. For example, extreme programming requires pair programming, something that would be hard for us to do and would drastically tank our productivity. Overall we made our decision based on what would work best for us as  team.
