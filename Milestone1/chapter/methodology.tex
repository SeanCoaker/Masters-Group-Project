\chapter{Methodology} \label{ch:methodology}

    Due to the nature of our team and project, We decided to implement an agile based development methodology.\ As a
    small team, it is easy for us to communicate with each other about the project, which allows us to be aware of what
    the other members of our team are doing, both of which are key factors for choosing an agile methodology.\ The
    question There are a variety of methodologies available, each with their own upsides and downsides, which are
    discussion below.\ Based on our research into these, we settled and agreed to use a scrum-based development
    methodology.\ Scrum is a combination of iterative and incremental development.\ Allowing for a best of both worlds
    approach of having early builds working, but also being agile, and able to add requirements during the development
    process\cite{srivastava_2017_scrum}.

    The scrum methodology is rather simple.\ To work effectively, it requires collaboration between an appointed ‘Scrum
    Master’ and the rest of the development team, as well as a ‘Product Owner’.\ The members will work in close
    collaboration with each other, and have multiple, continuous iterations of the software builds to create a finished
    product.\ The role of the scrum master is to eliminate impediments\cite{srivastava_2017_scrum}, and to create the
    conditions that allow the development team to work in their most effective manner, they take a leading role in
    choosing what sprints the team work on and are crucial when team splitting decisions are required.

    A typical scrum workflow consists of iterative scrum cycles, typically lasting one to three weeks each.\ Initially a
    requirements backlog is created.\ This is done in collaboration with the client and is the documentation that
    describes all the requirements that the software must meet.\ Essentially, it is a description of the product that the
    team are aiming to build.\ After the initial backlog is established, the scrum cycles can start.\ As before, each
    scrum cycle lasts typically up to three weeks.\ The cycle begins with the team, headed by the scrum master deciding
    which requirements to prioritise and implement in the upcoming cycle.\ The aim is to have a potentially
    shippable/working product at the end of each sprint, although this may not be possible immediately due to the
    workload required to build a framework, following sprints are expected to get increasingly more successful.\ Once the
    team have allocated the suitable requirements and resources to the sprint, the sprint begins. The team works on the
    project, meeting each day in what are known as ‘scrums’ to review progress made on each day and discuss any problems
    or insight they have found on the implementation. The team will also meet and hold a sprint review, with the product
    owner. This demonstrates the product and how it has been developed and the progress since the last sprint. The team
    will also review the previous sprint, to see if any changes need to be made in anticipation of the next
    one.\cite{nuevo_2011_scrumbased}

    For our project, we will be slightly modifying the implementation of scrum. The above-mentioned workflow is designed
    as a framework for full time software developers, which we are not. As students with other commitments, we will not
    be able to meet daily to discuss the project or work on it fulltime. It would be an unrealistic target to hold daily
    scrums, but we can still follow our adaptation of the methodology. We will aim to hold scrums at frequent intervals,
    most likely every week, other coursework dependant. This wil allow us to keep track of current sprints, as well as
    compensate for this by being realistic in what we can achieve in the given sprint timeframe, and what we set as the
    sprint’s goals.

    Ultimately, we chose to use scrum for a variety of reasons. Primarily because we realised that as a small team,
    frequent communication would be easy to achieve, so we would be able to make use of this framework in the most
    effective manner, compared to other methodologies. We would be able to meet frequently and use that to ensure
    progress is made on the project. Scrum works well with this, which enables us to work together to frequently get
    working, testable products and allow us to implement more features that would otherwise rely on this. We also feel
    that using scrum would allow us to have a higher quality product, as the frequent iterations during sprints would
    allow us to streamline our implementation and constantly improve it, which also minimises risk.

    In comparison, the waterfall methodology where implementation begin much later in the lifecycle once all
    documentation and test suites are formulated. Using scrum, we start programming earlier, giving us more time to deal
    with any difficulties we encounter. It allows us to identify problems earlier, have longer to address them, or
    encourage us to plan alternative implementations that avoid such development deadlocks.

    We chose scrum over other pure agile methodologies mainly because we thought it was the best fit for our team and
    situation. Other methodologies, like extreme programming, could have worked well, but didn’t suit our situation as
    students as well and added unnecessary risk to an already time critical project. For example, extreme programming
    requires pair programming, something that would be hard for us to do consistently and would drastically diminish our
    overall productivity.
