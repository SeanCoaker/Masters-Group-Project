% Two sided means the left and right margins are different sizes and they alternate every page.
% If your document is printed to be book or spiral bound this allows for a thick spine to not
% eat into the space for your page content.
\documentclass[11pt, a4paper, twoside, openright]{custard}

% All imports, packages, and configuration in here.
% Your document should be about content so we abstract away the styling rules and tools we are using.
%% Here you can specify new commands and environments that you intend
%% to use. Using commands can make your document easier to write, read
%% and be more consistent.

\usepackage[linesnumbered,ruled]{algorithm2e}
\DeclareMathOperator*{\argmin}{arg\,min}

\usepackage{diagbox}
\usepackage{appendix}
\usepackage{textcomp}
\usepackage{setspace}
%\usepackage[document]{ragged2e}
\usepackage{verbatim}
\usepackage{multirow}
\usepackage{multicol}
\usepackage{booktabs}
\usepackage{enumitem}
\sloppy
\usepackage{graphicx}
\usepackage{threeparttable}
\usepackage{epsfig}
\usepackage{epstopdf}
\usepackage{float}
\usepackage{enumitem}
\usepackage{cite}
\usepackage[export]{adjustbox}
\usepackage{algorithmic}
\usepackage[nohyperlinks,printonlyused]{acronym}
\usepackage{amsmath}
\usepackage{amsfonts}
\usepackage{array}
\usepackage{tabularx}
\usepackage{xltabular}
\usepackage{longtable}
\usepackage{times}
\usepackage{amssymb}
\usepackage{hhline}
\usepackage{color}
\usepackage{soul}
\usepackage{colortbl}
\definecolor{Gray}{gray}{0.85}
\usepackage{rotating}
\usepackage{fix2col}
\usepackage{pdflscape}
\usepackage{pdfpages}
\usepackage{stmaryrd}
\usepackage[export]{adjustbox}
\usepackage{bbm}
\usepackage{relsize}
\usepackage{xfrac}
\usepackage{bibentry}

%\usepackage{refcheck}

%watermarking
\usepackage[english]{babel}
\usepackage{tikz}

%% Uncomment the following line for hyper links - not recommended for printing
\usepackage[colorlinks, linkcolor=, anchorcolor=, citecolor=, filecolor=, menucolor=, runcolor=, urlcolor=]{hyperref}

\setcounter{tocdepth}{1}
%\setcounter{minitocdepth}{3}
\linespread{1.31}

\newcommand\litem[1]{\item{\bfseries #1:\enspace}}

\interdisplaylinepenalty=2500

\newcolumntype{L}[1]{>{\raggedright\let\newline\\\arraybackslash\hspace{0pt}}m{#1}}
\newcolumntype{C}[1]{>{\centering\let\newline\\\arraybackslash\hspace{0pt}}m{#1}}
\newcolumntype{R}[1]{>{\raggedleft\let\newline\\\arraybackslash\hspace{0pt}}m{#1}}

\renewcommand{\thefootnote}{\fnsymbol{footnote}}
\setlength{\LTpre}{-10pt}\setlength{\LTpost}{-30pt}%
\newcommand{\oiint}{\begin{picture}(0,0)(-10,-2)\put(0,0){\oval(12,8)}\end{picture}\iint}
\renewcommand{\mathbf }{\boldsymbol}

\def \eg{e.g.\ } % Allows you to write \eg in LaTeX instead of typing e.g. so that every single one will be formatted the same.
\def \Eg{E.g.\ } % Define some other common variants. If you later want to change one of these definitions,
\def \ie{i.e.\ } % it will update all the usages throughout the document.
\def \Dr{Dr.\ }
\def \vs{vs. }
\def \etal{\emph{et al.\ }}
\def \sota{state-of-the-art }
\def \handcrafted{hand-crafted }

\usepackage{listings,lstautogobble}
\usepackage{sourcecodepro}
\pdfmapfile{=SourceCodePro.map}
\lstset{
	xleftmargin=0.5cm,frame=tlbr,framesep=4pt,framerule=0.5pt,
	language=,
	upquote=true,
	columns=fixed,
	tabsize=2,
	extendedchars=true,
	breaklines=true,
	numbers=left,
	numbersep=10pt,
	basicstyle=\ttfamily\scriptsize,
	numberstyle=\tiny,
	stringstyle=\ttfamily,
	captionpos=b,
	showstringspaces=false,
	autogobble=true
}

\usepackage[font=small,skip=10pt]{caption} %,format=hang
%\usepackage[labelformat=simple]{subcaption}
\usepackage[labelformat=simple]{subfig}
%\captionsetup[figure]{format=hang}
%\captionsetup[lstlisting]{format=hang}
\renewcommand{\thesubfigure}{\Alph{subfigure}.}

\renewcommand{\thefootnote}{\arabic{footnote}}

% Use IEEEtran citation style.
\bibliographystyle{IEEEtran}

\def\samplefont#1{%
    % set font style and save name
    #1\edef\savedname{\fontname\font}%
    % print small sample
    {\leavevmode\tt\hbox to 1in{\savedname:\hss}}%
    abcxyz ABCXYZ 123\par
}


\begin{document}

% The custom data for Swansea University and your degree name.
% The \protect\\ command forces a new line in the title which might be otherwise overriden by the template
	\title{Walking Aid Reminder Device for Dementia Patients}
	\author{Bangor Health Clinic Group\protect\\}
	\awardinginst{Swansea University}
	% Comment / uncomment your degree type as needed.
	\degree{Masters of Engineering}

% Institution details and logo
	\department{Faculty of Science and Engineering}
	\university{Swansea University}
	\unilogo{graphics/swansea.png}

% Hard code the date or allow the LaTeX compiler to fill it in whenever you recompile the document.
	\date{\today}

% Build the title and declaration pages, and pad the document so the text starts on a right hand book page.
% Page numbering is in roman numerals until the first page of an actual chapter which resets numbers
% starting from 1 at that point.
	\frontmatter%
	\maketitle
	\cleardoublepage

% Optionally you can make a bank of known acronyms in acronyms.tex that you can call on throughout your document.
	%\input{acronyms}

% Reset numeric page numbering from page 1
	\mainmatter%

% Insert the code for each of your chapters
	\chapter{Topic} \label{ch:topic}

    This chapter details the problem and task our project aims to solve along with the limitations that could affect
    our project, including an analysis of the current solutions our project has to compete with.\ We will provide
    background research on dementia patients and senior citizens with similar conditions (hereby depicted as the
    'User(s)') and their issues with forgetting their walking aids, and how wearable devices can have a varying
    psychological impact on them.\ Finally, we will detail our current progress consisting of two initial meetings
    with our client, and our takeaways from these engagements.

    \section{Background}

        The client came to us with the idea of developing a product for those suffering with dementia, a syndrome that
        is usually associated with a declining functionality of the brain.\ Dementia can manifest itself with a large
        range of symptoms, most commonly including memory loss, loss of mental sharpness, or loss of the use of
        language~\cite{nhs_choices}.\ More importantly to our project, one other symptom can be a loss in movement
        skills, or an increased difficulty in moving.\ Dementia has been recognised as an illness that causes an
        increase in falls within
        patients~\cite{doorn_gruber-baldini_zimmerman_hebel_port_baumgarten_quinn_taler_may_magaziner_et_al._2003}.\ As
        dementia affects mainly more elderly population, falls are often more dangerous to them.\ It is therefor not
        uncommon for dementia patients to be using walking aides to help mitigate this, a walking aid is only effective
        if it is used by the patient, and as explained by the client, this is something the users may struggle with
        remembering to use

    \section{The Problem}
        Upon completion of our initial meeting with the client, we clarified the motivation behind this project and the
        problem that we are working together to solve.\ The problem is to develop a solution that detects when a
        user is moving without their walking aid and reminds the patient (with a recorded message by a
        friend or a relative) to take their walking aid with them.\ Initial discussions between ourselves and the client
        identified current issues with users feeling uncomfortable in being forced to wear foreign objects,
        meaning we would need to take this into account when developing our solution.\ We also clarified that users get easily alarmed and frightened by generic “obnoxious” alarms, often associating them with those
        notifying them of danger, such as fire alarms.\ The client suggested that we facilitate a recording feature
        within our solution that would allow recognisable voices to the user to remind them to use their
        walking aid.

        \subsection{Similar Solutions}
            Current solutions include the use of locator systems that allow users to easily track down valuables such as
            keys or a wallet.\ Such systems include the Tile ecosystem which allows a user to attach a Tile device to
            their valuables and then use a smartphone app to trigger an alarm from the Tile device that notifies the
            patient of the location of their valuable.\ As previously mentioned, users can get frightened and
            disorientated by the sounds of alarms often associating them with danger rendering these forms of solutions
            unsuitable for our problem.\ This is without considering how difficult a users mind finds navigating through
            a smartphone device to open an application and request their Tile device to ring an alarm to help them
            identify the location of their valuable.\ Another problem with such approaches, is the lack of functionality
            for sending an alarm to the patient if they start walking without the aide.\ Solutions such as Tile would
            help a patient find the aide if it had been misplaced but implementing functionality to remind the patient
            to find the aide if they have already started walking without it, is lacking entirely.\ Other more legacy
            systems that carers may use to notify themselves that their patient is moving include hanging items from
            door frames that clatter together when the patient walks through the door or adding pressure pads under door
            mats that sound an alarm when the patient steps on the door mat.\ Our solution is focussing on the protection
            of users that are alone and wanting to move around their home or ward, meaning that door mat pressure pad
            based solutions and methods for alarming a carer would be insufficient.\ We also would like our design to be
            more elegant than those rudimentary solutions, if the patient is walking with the aide, they will still get
            an alarm from the pressure pad.\ A well implemented system could alleviate both those problems, and increase
            the quality of life of the patient and facilitate the carer.

        \subsection{Limitations}
            Our main limitation for our project is that we need to develop a discrete device that will not make the
            users feel uncomfortable in any way and will minimise discomfort to an extent where it is
            acceptable for day-to-day life.\ Early plans for the device lean towards a watch style device that the
            patients can wear on their wrist to track their movement.\ If we were to create a wrist wearable device
            then we would need to ensure that the footprint of device is small enough to be worn comfortably.\ This limits
            the hardware that we can feasibly use for our project.\ We also need to consider the hardware being
            used and how they can all be fit within a wrist device.\ The head of Swansea University’s Embedded Systems
            module has kindly offered to supply us with ESP32 based TinyPICO devices which would be suitable for this
            project due to their small form factor.\ We would need to consider how an accelerometer could be attached to
            the TinyPICO to allow both devices to fit within a watch casing.\ If we could find hardware small
            enough to build into a watch-based prototype, we may still run into issues with the patients wanting to use
            the product.\ During our meeting we were told that patients already had to wear wrist tags or similar items.
            Adding more items the patient needs to wear is unlikely to be well received.\ Our form factor could take the
            shape of something that clips onto what the patient is already wearing, such as clothes or belts, or a tag
            that they already use.

            Other limitations for the wrist device are that it should not contain any strong LEDs, obnoxious vibration
            motors or alarms to avoid startling the user.\ Avoiding the use of LEDs would be of benefit to us here as it
            would allow the watch device to save battery during operation.\ Another limitation to the watch device is
            that it would need to be power efficient and avoid the users needing to frequently charge the device.\ The
            patient needing to do this would work against our goal of creating a user-friendly experience for them.\ The
            ESP32 chips included on the TinyPICO boards utilise a facility called 'deep sleep' or hibernation states,
            which effectively powers down certain modules connected to the board.\ We could create a system here that
            fires an interrupt when the accelerometer detects movement, then forcing the ESP32 to wake up and handle the
            interrupt.\ The device could therefore be in sleep whilst the patient is static, to save battery.

            The device attached to the walking aid has much less limitations, and so we do not need to consider such a
            small form factor as it is not being worn but will still be using a TinyPICO to encourage compatibility and
            interchangeable/modular development experiences.\ Our limitation with this device is to also disable the use
            of LEDs to avoid startling the patients, and to include a speaker and microphone to allow a relative or
            carer to record a voice note which will be played to remind the patient to take their walking aid with them
            when moving.\ The TinyPICO boards include a very minimal amount of storage space and so we may need to
            include a SD card to store the recordings.\ We will also be limited to the budget of £150 that we have been
            assigned and must ensure that all the devices needed to build the system can be purchased within our budget.

    \section{Current Work}
        As stated earlier in this chapter, we have held initial meetings with the client where we have clarified the
        problem they aim to solve with this project and outlined the project scope.\ On the 18th of November we held an
        introductory meeting with the client where we gained an understanding of what the problem is and what kind of
        system the client was expecting to be produced.\ We clarified that we would need to gain our supply of hardware
        ourselves and that a budget of £150 would be allocated to us to aid with the procurement of the necessary
        hardware devices.\ However, within this initial meeting we failed to identify the final direction the client
        wanted the project to head down and instead came away with the option of either developing a wearable device
        that would detect when the users was moving, or to use a non-wearable device such as a pressure
        blanket that would detect when the patient had got up from where they were static.\ We agreed with the client
        that we would schedule a second meeting for the 25th of November and within that time analyse the advantages
        and disadvantages to each method of developing the solution.\ We then agreed that we would return with a solution
        that we thought would best suit the design brief and that would best suit the development talent available to us
        within our team.

        Within our own intra-team meeting we decided upon building and developing a wearable device solution over a
        non-wearable solution due to the extra features that could be included into a wearable device such as a fall
        detection system, a system recommended to be included by the client.\ We felt that despite a non-wearable device
        being a plausible route to take the project down, that factors such as a users moving off a pressure
        pad without actually standing up and walking would diminish the effectiveness of our solution.

        On the 25th of November we hosted our second meeting with the client and established the team's preferred route
        for the development of this project.\ The client was content with this and agreed that the solution should be
        developed as a wearable device.\ We finalised the £150 budget with the client and agreed that our next steps
        would be to complete our milestone 1 document, including user requirements, and compiling a list of necessary
        hardware to develop the project.\ Our next meeting with the client is scheduled for December 16th where we will
        finalise the user requirements and compile a list of hardware to be purchased with the budget made available to
        us.

    \section{Project Aims}
        For evaluation purposes at the conclusion of this project, we will detail within this section a list of aims
        that should be met along with lower level objectives that define a set of criteria that will allow use to meet
        said aims.\ An evaluation within milestone 3 will look to compare the final product produced against the aims set
        out in this section in an attempt to gauge how successful our project is.\ A list of our aims and their lower
        level objectives can be seen below.

        \begin{itemize}
            \item Develop a solution that reminds the user(s) to take their walking aid with them when attempting to walk.
            \begin{itemize}
                \item We should create a wearable device that includes a tri axial accelerometer to detect when the
                    user(s) have started moving.
                \item A device should be connected to the walking aid that also contains tri axial accelerometer that
                    will be used to detect if the walking aid is moving whilst the wearable device is moving.
                \item The walking aid device should contain a microphone and speaker that allows a carer to record a
                    reminder for the user(s) in attempt to avoid startling the user(s), which could happen when using
                    generic alarms.
            \end{itemize}
            \item Create prototype devices that avoid startling the user(s) and avoids making them uncomfortable.
            \begin{itemize}
                \item The wearable device should be designed and built such that it is inconspicuous in attempt
                    to not draw attention to it from the user(s).
                \item Unless the user(s) is deaf, we should avoid the use of LEDs, vibration motors, and the use
                    of generic alarms.
                \item The wearable device should be developed to be worn on the wrist rather than on a more
                    uncomfortable body part such as the neck.
            \end{itemize}
            \item Produce a solution that the client concludes is satisfactory.
            \begin{itemize}
                \item Produce a document of user requirements and receive confirmation from the
                    client that the requirements are sufficient.
                \item Allow for changes in requirements during project
                    development.\ Our chosen agile methodology will allow for easy integration of changed user
                    requirements here.
                \item Develop the solution such that it is compliant with the user requirements.
                    \end{itemize}
        \end{itemize}
	\chapter{Project Definition}
\label{chap:proj-def}

In this chapter, we'll go through the Project in further depth. There will be a brief introduction to the Project's goals and services, as well as the solution that demonstrates how the entire system will operate. The evaluations that will be done will justify if the project functions normally, thus we will outline how we will assess the project.

\section{Context}

The challenge of this project is to develop an effective system that interacts with users, who are dementia patients. The reason we need to make an app like this is that dementia patients must be reminded to take their walking aid with them.


\section{Description of the Project}

This project will support the development of a wearable device including an application. The dementia patient or user will be encouraged to use the walking assistance in this application. A recorded message from a family member or relative will act as the notification. The device will detect if the user is sleeping and only then activate, since we do not want the device to begin notifying the user while he is sleeping, as the user might be startled by the notice.

\section{Project Solution}

\section{Project Aims}

\section{Project Success and Evaluation Criteria}

	\chapter{Requirements} \label{ch:Requirements}

In this section, we will detail the project's functional and non-functional requirements, which are broken down into
higher-level user requirements as well as lower-level specifications that will describe the process our team will go
through to ensure the user requirements are met. These written requirements have been demonstrated to the client and they have provided their consent for the project to be based upon them.

\section{Functional Requirements} \label{sec:func_requirements}

\vspace{1em} \small
	\begin{xltabular}[H]{\textwidth}{c | X | X | X}
		\caption[Functional Requirements.]{A table of functional requirements split into user requirements and their relevant specifications needed to meet those user requirements, along with the progression made so far.}\\

		\toprule

		Code & User Requirement & Specification & Progress\\

		\midrule
		\endfirsthead

		\toprule

		Code & User Requirement & Specification & Progress\\

		\midrule
		\endhead

		\hline
		\multicolumn{4}{|r|}{{Continued on next page}}\\
		\hline
		\endfoot

		\bottomrule
		\endlastfoot

        FREQ1

        &

        The wearable device should detect when a patient has walked more than 1 metre before communicating with the walking aid.

        &

        We can use a tri axial accelerometer to detect changes in acceleration that are indicative of the user moving or being mobile. Once movement is confirmed, we will then commuicate with the walking aid device to ensure the user has successively reached and engaged with it, prior to alerting them to use it.

        &

        -\\

        \midrule

        FREQ2

        &

        Patients should be alerted with the voice of a friend, carer or relative to avoid startling them.

        &

        The device to be attached to the walking aid should include a microphone and speaker that will allow the user to record a voice note and store it on the device. We may need to include an SD card within this device that will store the voice note if need be.

        &

        -\\

        \midrule

        FREQ3

        &

        The wearable device should include a solution for deaf people that still reminds them to take their walking aid with them without the need for an audio alarm.

        &

        The wearable device could use a vibration motor here that vibrates to remind the user to use their walking aid. We can also utilise the LEDs on board the TinyPICOs to flash to remind the user also. There are issues here with potentially startling the patient with the use of vibration and LEDs, however we feel this is most feasible method for meeting this user requirement.

        &

        -\\

        \midrule

        FREQ4

        &

        If development time allows, the system should include fall detection as a stretch goal feature.

        &

        Using the tri axial accelerometer mentioned in the specification of FREQ1, we could detect acceleration and movement along the negative side of the y-axis in attempt to detect when the patient has fallen. An alert system can be used in accordance to alert a nearby carer or relative.

        &

        -\\

        \midrule

        FREQ5

        &

        The wearable device should communicate to the walking aid device to let it know when it's started moving.

        &

        To meet this requirement we investigate the use of 433MHz Rx/Tx modules for low power and low level communication between the 2 devices in the system, this technology should allow for the basic level of communication required, with minimal power use and minimal complexity.

        &

        -\\

	\end{xltabular}
	\label{tbl:func_reqs_table}
 \vspace{5em}

\section{Non-Functional Requirements} \label{sec:non_func_requirements}

\vspace{1em} \small
	\begin{xltabular}[H]{\textwidth}{c | X | X | X}
		\caption[Non-Functional Requirements.]{A table of non-functional requirements split into user requirements and their relevant specifications needed to meet those user requirements.}\\

		\toprule

		Code & User Requirement & Specification & Progress\\

		\midrule
		\endfirsthead

		\toprule

		Code & User Requirement & Specification & Progress\\

		\midrule
		\endhead

		\hline
		\multicolumn{4}{|r|}{{Continued on next page}}\\
		\hline
		\endfoot

		\bottomrule
		\endlastfoot

        NONFREQ1

        &

        The watch should be a small enough form factor to fit on the wrist of the patient.

        &

        Deciding to use TinyPICO devices as the main board of the device will allow us to keep the device to a small form factor given the TinyPICO is 18mm x 32mm. We will also take into account the form factor when deciding upon extra hardware to add to the devices.

        &

        -\\

        \midrule

        NONFREQ2

        &

        The devices shall be power efficient to avoid the patient needing to charge them often.

        &

        The TinyPICO devices we will use as the main boards for the devices include an ESP32 chip capable of using deep sleep cycles. These cycles allow the ESP32 to power down non critical components in order to save power. We can create an interrupt within the code here that powers the devices on when an alarm needs to be fired due to the patient moving. This means that the devices will only need to be fully powered on when movement is detected.

        &

        -\\

        \midrule

        NONFREQ3

        &

        The devices shall avoid startling the patients with the use of LEDs and vibrations unless they are deaf.

        &

        In this case we would power down the LEDs and Vibration motor at all times to avoid startling the patient. Powering down these devices will also allow us to save battery.

        &

        -\\

        \midrule

        NONFREQ4

        &

        The wearable device should be discrete enough that it does not make patients uncomfortable wearing it.

        &

        We intend to design the device to make it as close the design of a watch as possible, keeping it small and sleek so that it looks like a fashion accessory rather than a medical device. Using small hardware devices we can keep a small form factor so that the device is not overly noticeable on the patient's wrist.

        &

        -\\

        \midrule

        NONFREQ5

        &

        Security of devices should prohibit outside devices from communicating with the network.

        &

        A possibility here is using an agreed upon 'sync word' between our 2 devices that only reads communcations from devices using the same 'sync word'. This would stop other devices being able to communicate with the network unless they knew the sync word being used.

        &

        -\\

		\midrule

        NONFREQ6

        &

        The developed prototype wearable device shall be designed such that it can fit various wrist dimensions.

        &

        The strap of the wearable device should be designed such that it can be adjusted in a similar way to how normal watches can be adjusted.

        &

        -\\

		\midrule

        NONFREQ7

        &

        The coding system the devices run on should be efficient enough to react to real time actions.

        &

        Our code should be developed with efficiency in mind that will allow for real time readings from the accelerometer to precisely detect when movement has passed 1 metre in order to notify the walking aid device that this has happened. The Rx/Tx communication module will allow for fast communication between the devices to ensure that user is promptly reminded to take their walking aid with them.

        &

        -\\

		\midrule

        NONFREQ8

        &

        The system should be delivered upon its conclusion with relevant documentation, including a user manual.

        &

        Our milestone 3 document will include the necessary documentation needed to be handed over to the client upon the handover of our project. This milestone will include a user manual along with any designs and testing documentation that were created during the development of the product.

        &

        -\\

	\end{xltabular}
	\label{tbl:non_func_reqs_table}
 \vspace{5em}


% Formatting citations properly when they are rendering incorrectly in your PDF can be fiddly,
% espectially when punctuation and international characters are involed. Sometimes multiple re-compilations
% are needed in addition to clearing temporary auxiliary files to see your changes in your document.
% Insert the bibliography using citations contained in the file citations.bib
	\bibintoc
	\bibliography{citations}

\end{document}
