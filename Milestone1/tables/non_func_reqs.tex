\small
	\begin{xltabular}[H]{\textwidth}{c | X | X}
		\caption[Non-Functional Requirements.]{A table of non-functional requirements split into user requirements and their relevant specifications needed to meet those user requirements.}\\

		\toprule

		Code & User Requirement & Specification\\

		\midrule
		\endfirsthead

		\toprule

		Code & User Requirement & Specification\\

		\midrule
		\endhead

		\hline
		\multicolumn{3}{|r|}{{Continued on next page}}\\
		\hline
		\endfoot

		\bottomrule
		\endlastfoot

        NONFREQ1

        &

        The watch should be a small enough form factor to fit on the wrist of the patient.

        &

        Deciding to use TinyPICO devices as the main board of the device will allow us to keep the device to a small form factor given the TinyPICO is 18mm x 32mm. We will also take into account the form factor when deciding upon extra hardware to add to the devices.\\

        \midrule

        NONFREQ2

        &

        The devices shall be power efficient to avoid the patient needing to charge them often.

        &

        The TinyPICO devices we will use as the main boards for the devices include an ESP32 chip capable of using deep sleep cycles. These cycles allow the ESP32 to power down non critical components in order to save power. We can create an interrupt within the code here that powers the devices on when an alarm needs to be fired due to the patient moving. This means that the devices will only need to be fully powered on when movement is detected.\\

        \midrule

        NONFREQ3

        &

        The devices shall avoid startling the patients with the use of LEDs and vibrations unless they are deaf.

        &

        In this case we would power down the LEDs and Vibration motor at all times to avoid startling the patient. Powering down these devices will also allow us to save battery.\\

        \midrule

        NONFREQ4

        &

        The wearable device should be discrete enough that it does not make patients uncomfortable wearing it.

        &

        We intend to design the device to make it as close the design of a watch as possible, keeping it small and sleek so that it looks like a fashion accessory rather than a medical device. Using small hardware devices we can keep a small form factor so that the device is not overly noticeable on the patient's wrist.\\

        \midrule

        NONFREQ5

        &

        Security of devices should prohibit outside devices from communicating with the network.

        &

        A possibility here is using an agreed upon 'sync word' between our 2 devices that only reads communcations from devices using the same 'sync word'. This would stop other devices being able to communicate with the network unless they knew the sync word being used.\\

	\end{xltabular}
	\label{tbl:non_func_reqs_table}
