\small
	\begin{xltabular}[H]{1.4\textwidth}{p{0.2\textwidth} | p{0.5\textwidth} | p{0.8\textwidth}}
		\caption[Non-Functional Requirements.]{A table of non-functional requirements split into user requirements and the progress that has been made so far to meet those requirements.}\\

		\toprule

		Code & User Requirement & Progress\\

		\midrule
		\endfirsthead

		\toprule

		Code & User Requirement & Progress\\

		\midrule
		\endhead

		\hline
		\multicolumn{3}{|r|}{{Continued on next page}}\\
		\hline
		\endfoot

		\bottomrule
		\endlastfoot

        NONFREQ1

        &

        The wearable device should be a small enough form factor to be comfortably worn by the patient.

        &

        Our CAD designs in section \ref{sec:cad} demonstrate the minimal form factor that we are able to develop for our devices. This demonstrates that we can adequately meet this requirement.\\

        \midrule

        NONFREQ2

        &

        The devices shall be power efficient to avoid the patient needing to charge them often.

        &

        This requirement will be in progress throughout the development of the project. We have partly implemented the solution to this by selecting low powered hardware and low powered technology for communication between our devices. We will explore ESP32s sleep modes in future in an attempt to try and lower power consumption further.\\

        \midrule

        NONFREQ3

        &

        The devices shall avoid startling the patients with the use of LEDs and vibrations unless they are deaf.

        &

        This is easily implementable and is still in progress only because we are currently using LEDs for the simulation of the reminder system. Upon the completion of the system, we will only use a recorded audio file for the reminder system, or the vibration motor if the patient is deaf.\\

        \midrule

        NONFREQ4

        &

        The wearable device should be discrete enough that it does not make patients uncomfortable wearing it.

        &

        Our CAD designs in section \ref{sec:cad} demonstrate how discrete our devices will be. We have selected our hardware components with this requirement in mind and feel that we have met this requirement based upon our prototypes.\\

        \midrule

        NONFREQ5

        &

        Security of devices should prohibit outside devices from communicating with the network.

        &

        We have met this requirement with the use of ESP-Now for communication between our devices. ESP-Now utilises Primary Master Key along with multiple Local Master Keys for its communication security \cite{esp}.\\

		\midrule

        NONFREQ6

        &

        The developed prototype wearable device shall be designed such that it can fit various limb/walking aid dimensions.

        &

        As mentioned in section \ref{sec:cad}, we have created CAD designs and a prototype with adjustability for being able to fit various walking aids/limb sizes.\\

		\midrule

        NONFREQ7

        &

        The coding system the devices run on should be efficient enough to react to real time actions.

        &

        This requirement is still in progress but, we have made advancements to it. We have done this by already limiting the monitoring our accelerometers perform to only monitor the data we need, as we did in listing \ref{lst:spi_setup}. Future progression on this requirement will include cleaning up code as much as possible to limit memory usage, as well as disabling unnecessary hardware within our TinyPICO board such as the Bluetooth technology if we choose not to use it.\\

		\midrule

        NONFREQ8

        &

        The system should be delivered upon its conclusion with relevant documentation, including a user manual.

        &

        We have a deadline to submit our user manual within our Milestone 3 submission on the 26th of April. Our user manual will be complete by then.\\

	\end{xltabular}
	\label{tbl:non_func_reqs_table}
