\small
		\begin{xltabular}[H]{\textwidth}{p{0.47\textwidth} | p{0.47\textwidth}}
			\caption[BLE Advantages and Disadvantages]{A table listing the advantages and disadvantages of the BLE protocol.}\\

			\toprule

		 	Advantages & Disadvantages\\

			\midrule
			\endfirsthead

			\toprule

			Advantages & Disadvantages\\

			\midrule
			\endhead

			\hline
			\multicolumn{2}{|r|}{{Continued on next page}}\\
			\hline
			\endfoot

			\bottomrule
			\endlastfoot

			Offers low energy consumption \cite{ble_adv_dis} allowing us to run the system off battery power for a long time without the patient needing to remember to change the batteries in our device.
			
			&
			
			It cannot operate at the higher data transfer rates that are offered by WiFi and mobile cellular technologies \cite{ble_adv_dis}.\\
			
			\midrule
			
			Uses a maximum message size of 255 bytes \cite{ble_adv_dis}. This would be perfect for our project as it would minimise communication overheads as we are only sending notification messages between the devices.
			
			&
			
			As with all wireless communication, it is susceptible to being compromised by a malicious attacker. For our project this may allow the attacker to repeatedly call a function that plays the reminder alarm from the walking stick device.\\
			
			\midrule
			
			As BLE includes Bluetooth technology, any board that includes BLE technology within their chips will be able to communicate with each other \cite{ble_adv_dis}. This would mean that it may be easier to upscale our production of the devices in future.
			
			&
			
			\\
			
			\midrule
			
			"BLE devices are robust to operate in congested environment due to introduction of V5.0" \cite{ble_adv_dis}. This will provide huge benefit should our devices be used in busy environments such as medical centres and care homes.
			
			&
			
			\\

		\end{xltabular} 
		\label{tbl:ble}