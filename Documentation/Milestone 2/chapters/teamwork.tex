\chapter{Team Work}

Once we started to predict that we may experience delays with the project, we ensured that we ordered enough hardware that would give us multiple sets of walking and wearable devices. This would allow our team to each be able to access devices and develop software from home. The ordering and taking delivery of the hardware was handled by our client, but the frequent communication between our team and the client regarding what hardware to order, budget queries and the collection of our hardware was handled by our communications officer Sean. The team all decided that we should utilise TinyPICOs for both the walking aid and wearable devices due to our previous experience of software development with them. Pedro then played a vital role in the selection of the remaining hardware due to his vast experience and knowledge of working with embedded systems.

The current progress within code development was completed by Sean, due to his recent experience in working with Arduino embedded systems, especially in the field of communication within networks of embedded systems. He developed the initial communication system between our TinyPICO devices, before moving on to develop the current solutions of walking detection, which included the research into and development of two solutions utilising our ADXL accelerometers. Each team member in future will assume the role of developer, allowing for tasks to be split to ensure that they are completed on time and to a high standard. 

Due to the previously mentioned vast experience that Pedro holds within this field, he assumed the role of developing CAd designs, as well as 3D printing the prototype housing for our devices. This required him to look into the dimensions of walking aids in order to create solutions that could be fixed to the walking aids in future.

All team members assumed the role of documentation developers, with Sean writing up the introduction section and completing the research into background work relating to our project. He also completed the write up of the majority of the current progress section which included research into the differences between communication protocols, as well as discussions on what software development he has carried out so far. Pedro contributed to the current progress section of this document with his insight into what progress has been made in terms of CAD designs and prototype development. Panayiotis built on his scheduling work from Milestone 1 and redeveloped our schedule for this Milestone to recognise the time we have lost due to delays, as well as to demonstrate why we are still confident of completing this project on time and how we aim to go about doing that. Panayiotis also worked with Sean to provide details within this document of how each requirement has progressed so far. Finally, Matthew worked on the development of the updated risk analysis section of this document, outlining the risks we had already faced, as well as adding any new risks to our previous list and updating our mitigation strategies. He also outlined any changes to our analysis of the likelihood of risks occurring as well as our analysis of the severity of the consequences to each risk. In future, all team members will continue their assigned roles of equally developing documentation to ensure that a high quality piece of work is handed in for Milestone 3 as well as ensuring that an adequate user manual is submitted to the client. 