\chapter{Risk Analysis} \label{ch:Requirements}

In this section, we have completed a review and analysis of the risks from our Milestone 1 document \cite{coaker}. Where necessary, we will detail where we have run into risks, and if so, what steps we took to manage them. We will also detail how progress in the project has affected the our risks and how the risks may change moving forwards.

\section{Reflection of Risk since Milestone 1}
Of the risks detailed in Milestone 1 \cite{coaker}, none of them came to pass in a way that affected our project. We did however encounter certain situations that in hindsight should have been included within the original set of risks that we outlined in Milestone 1. These risks will detailed shortly, but they both share the same general theme; issues with the procurement of hardware for the project. 

The first stumbling block we ran into was with the total price of the hardware we were requiring being greater than the budget we had been allocated for the project. This should have been a relatively simple risk within the risk assessment that unfortunately we had overlooked. Luckily, we were able to use alternative methods to procure the hardware meaning that the lack of budget did not have a significant effect on the choice or quality of hardware we are using. As mentioned previously in this document, we ended up borrowing two TinyPICOs from the university and using our own funds to purchase the remaining hardware that we need. 

In a similar vein, we encountered another issue with the procurement of our hardware. This time with regards to the time taken for delivery. This was also a slight oversight with regards to our initial time plan as well. The cause of this issue was directly related to the previous issue mentioned, budgetary constraints. With the hardware within the budget being procured via the university, we ran into delays that we hadn't expected and planned around. As previously mentioned in this document, the hardware took four weeks to arrive, due to being ordered through the university's system, the immediate effect of that being the loss of development time we encountered as a result. 

In terms of risk assessment though, these issues do does have a knock on effect on the project going forwards. Multiple of our proposed risks detailed within Milestone 1\cite{coaker} (RSK1, RSK3, RSK8) detail the potential for damage to, or the need to replace hardware. Parts of the mitigation strategies in place for these risks involves having spare room in the budget to replace any faulty or broken hardware as required. As a consequence of the events mentioned prior, we no longer have the space within the budget that allows us to freely replace hardware if required without potentially significant personal financial expense. Compounding this, if we do need to procure additional hardware we may have to deal with the potential for another significant delay. As the university has already made arrangements for the purchasing of the hardware once, we can presume that the delay wont be the same four weeks we initially faced. However, we cannot be sure of this, meaning there is potential for an unknown delay that would be difficult to plan around. This becomes more of a risk as time progresses and our final deadline draws closer. If a crucial piece of hardware was to break with less than a month before the final deadline, it would be extraordinarily difficult to manage.

\section{Amendments to our Risk Assessment}
With this in mind and forethought to what we might encounter further on within the project, we have chosen to revise our risk assessment to ensure we are fully aware of, and prepared to deal with, any risks in the project moving forwards. 

\paragraph{New and Amended Risk}
With the issues previously mentioned in mind, we have decided to extend the risk table to include them. It is important now that we are prepared in the case of hardware failure. In the event of failure the part will need replacing, presumably at short notice. Therefore it is imperative that we have are aware of the correct process of obtaining replacement hardware and agree on how it shall be funded if our budget can no longer cover it. We will also endeavour to avoid making committing changes, (e.g. soldering) until we are one hundred percent confident in what we are doing and have tested the components together first via other means (e.g. on a breadboard). Integrating these management solutions into our workflow will minimise the risk of accidentally "bricking" the hardware we are using.

Since the delay in ordering components resulting in us losing development time for the project, there is now a risk of insufficient time available for the adequate completion of the project. This is relatively straightforwards to manage however; frequent and effective communication, as well as regular organised meetings to discuss and manage progress on the project should ensure that we are familiar with the status of the project and whether or not we are falling behind schedule. 

Some of our risks identified in Milestone 1 need adjustment in the mitigation strategies as time remaining on the project diminishes. The mitigation strategies obviously vary depending on the risk, but the general theme is to ensure we are aware of the immediate actions to limit the damage of the risk and rectify it in as little time as possible. For example, in the event of a replacing hardware, we should already be aware of the correct channels to purchase replacements from as well as how the replacements are to be funded. This is important to know before the risk occurs, so that our response can be instantaneous. For the sake of succinctness, we will not all such changes in their entirety here, but will highlight the changes in the next section in the document. 

\section{Revised Risk Assessment}

\paragraph{Risks}
Based on our reflections and new amendments to our perceptions of risk within the project, we have made changes to our risks and risk management strategies. We have included the risk table, highlighting any amendments from Milestone 1 in bold script.

\vspace{3em} \small
\begin{xltabular}[H]{\textwidth}{c | X | X}
    \caption[Risks Table]{A table of risks along with strategies to mitigate those risks.}\\

    \toprule

    Code & Risk & Mitigation\\

    \midrule
    \endfirsthead

    \toprule

    Code & Risk & Mitigation\\

    \midrule
    \endhead

    \hline
    \multicolumn{3}{|r|}{{Continued on next page}}\\
    \hline
    \endfoot

    \bottomrule
    \endlastfoot

    RSK1

    &

    Our hardware devices may fail and will limit development and testing.

    &

    To mitigate this risk we will choose to use low cost but still effective hardware, which will allow for extra funds within our £150 budget should we need it to replace hardware during development. We also have the opportunity to attain TinyPICO devices from the University for this project, allowing us to minimise the effects on our budget. \textbf{The processes for ordering and funding replacement hardware should be known in advance of the event of any piece of hardware being broken. }\\

    \midrule

    RSK2

    &

    The Rx/Tx modules could fail disabling communication between the wearable and walking aid devices.

    &

    The replacement of these modules should not be much of an issue due to their low cost. The real issue would arise when an Rx/Tx module fails whilst in operation for a patient. We would need to form a protocol here that detects when communication is unable to occur between the 2 devices, and can alert the patient's carer of this.\\

    \midrule

    RSK3

    &

    Uploading erroneous code to our TinyPICO devices could brick the TinyPICO devices.

    &

    Should this occur, we could attempt to reflash previosuly working code onto the TinyPICO. If this fails, we would be left with the occurrence of risk RSK1 where we would need to replace the TinyPICO devices that have been bricked. The low cost of the TinyPICO devices should enable us to purchase some replacements if need be.\\

    \midrule

    RSK4

    &

    GitHub experience a malicious attack or a server failure which could cause our repository to be lost.

    &

    Mitigating this risk is difficult. It's unlikely this will happen and that we would lose our repository as GitHub likely uses a vast backup storage solution. But, should it happen it would be catastrophic and so we should mitigate against this risk. To do this, each developer within the team will store a clone of the repository on their personal system and the team will be able to piece the code back together should this risk arise.\\

    \midrule

    RSK5

    &

    The user requirements we accept could be too large to implement within the given time frame for the project, potentially leaving the client disappointed at project handover.

    &

    To mitigate against this risk, we have discussed our user requirements with the clients and feel that we have decided upon a set of features that we feel we can confidently implement within the time frame given to us. We have also some marked some features as stretch goals to ensure that the most important features are added first. \textbf{There will be frequent communication between group members to ensure everyone is aware of progress and any delays that might affect feature implementation.}\\

    \midrule

    RSK6

    &

    University commitments could impact the development and testing of the product.

    &

    We have attempted to account for this within our schedule by allowing for slippage. Slippage time could allow for the development of unfinished features. We could also spread unfinished work between developers as extra work in an attempt to complete the development of features on time. \textbf{Testing should be planned in advance when group members have time, so that when the testing needs to be carried out, it can be done quickly and efficiently. We are planning to run integration testing now throughout the remaining development of the project.}\\

    \midrule

    RSK7

    &

    The hardware we choose to use may lack the libraries and compatibility with other hardware for quality feature development.

    &

    To avoid this, we will select hardware that is compatible with the Arduino ecosystem to ensure that they are compatible with each other and that libraries are available for code development.\\

    \midrule

    RSK8

    &

    Natural disasters could bring about the loss of hardware and software being used for the development of our product.

    &

    The use of low cost hardware in this system will allow us to replace any compromised hardware if need be. We have decided to store the team's code in a GitHub repository which will allow our code to be protected in an off cite facility. The loss of developer computer systems is by far the biggest risk here, as our budget would not be able to cover the replacement of such computer systems.\\

    \midrule

    RSK9

    &

    Especially in the current coronavirus climate, our developers may be unwell for a period of time that has a negative impact on the development of the project.

    &

    Within our schedule we have included time for slippage that should allow for any time needed by the team to be taken off due to illness. Should a developer need to self isolate and should they not be experiencing symptoms, they could continue to work on the product from a remote location using the GitHub repository.\\

    \midrule

    RSK10

    &

    An Inadequate testing strategy could allow unidentified bugs to be released in the product when handing it over to the client. This could lead to a disappointed client.

    &

    To mitigate against this risk we have devised a testing strategy that ensures thorough testing is carried out throughout the development of our product. We are confident that the proposed testing strategy will allow the team to identify and correct errors in the system before the product is released to the client. \\

    \midrule

    RSK11

    &

    Poorly developed code could mean that despite substantial testing, many bugs could still be included in the product at the conclusion of the project lifecycle, leaving our product to be ineffective for the uses that the client requires.

    &

    To mitigate this risk, we will be following our testing strategy outlined in this document to allow for integration testing. This means that testing will take place every time a new feature is added to the system, allowing us to detect bugs quickly as features are implemented. \textbf{Code should be reviewed by another group member to aid in spotting any major bugs or errors.}\\

    \midrule

    RSK12

    &

    The wearable device we create may cause patients to feel uncomfortable limiting their use of the device.

    &

    We are slightly out of control with this risk as it mainly depends on how the patient reacts to the wearable. Having said this, we aim to make the wearable as discrete and as watch like as possible in an attempt to avoid the patient feeling uncomfortable when wearing it.\\

    \midrule

    RSK13

    &

    Our device could startle or scare the patient putting them in danger.

    &

    We have taken on board the advice of the client for this risk and will be ensuring the watch does not use vibrations or LEDs unless the patient is deaf. We will ensure that generic alarms are not used also in an attempt to avoid startling the patient.\\

    \midrule

    RSK14

    &

    One of our developers could leave the institution, lowering the number of developers we have available to work on the project.

    &

    Should this occur, we would definitely need to use the slippage time we have allowed for in our schedule. We would not be able to bring another developer into the team and would therefore need to share the workload of the developer that is leaving to the other developers on the team. \textbf{We will ensure that all of the group are familiar with what each other is doing, so that in the need to take over from a member on short notice, the transition will be quick, and seamless.}\\
    
    \midrule

    \textbf{RSK15}

    &

    \textbf{Delays in ordering replacement hardware could cause the project to fall irreparably behind schedule.}

    &

   \textbf{Steps to mitigate RSK1 and RSK3 should be followed. To minimise chances of "bricking" hardware, permanent, difficult to undo processes should only be completed when necessary and the system relying on those parts has been properly tested to ensure no damage will occur. We have also already acquired multiple sets of devices to ensure that we have fallback hardware if necessary.} \\
   
    \midrule

    \textbf{RSK16}

    &

    \textbf{Due to previous delays, insufficient time to adequately complete the project.}

    &

   \textbf{Frequent communications and meetings should be held to ensure the group are familiar with the current status of the project, with weekly meetings already being booked, and how well the project is following the agreed upon time plan. In the event of falling behind irreparably, communication with the client is important to ensure that the product produced is as useful as possible.}\\

\end{xltabular}
\label{tbl:risk_table}
 \vspace{6em}

\paragraph{Risk Matrix}
As a whole, its important to realise now that with time on the project diminishing, the impacts of risks we may encounter is only going to increase. We will have less time to manage or action the mitigation plan. Luckily within our time plan we allocated a certain amount of slippage time that we can use to deal with any risks we would encounter. Unfortunately, we have already been forced to use some of this time as a result in the delay in procuring hardware. We are confident that we have managed to recoup some of that time, but there is no doubt that we do not have as much now as when we started. Within the updated risk matrix below, an increase in impact is marked in italic. New risks have been marked in bold.


\vspace{3em} \small
\begin{xltabular}[H]{\textwidth}{X | X | X | X}
    \caption[Risk likelihood/impact matrix]{A matrix detailing the likelihood of a risk occurring along with the relative impact caused by that risk occurring.}\\

    \toprule

    \diagbox[innerwidth=3.2cm]{Likelihood}{Impact} & Low & Medium & High\\

    \midrule
    \endfirsthead

    \toprule

    \diagbox[innerwidth=3.2cm]{Likelihood}{Impact} & Low & Medium & High\\

    \midrule
    \endhead

    \hline
    \multicolumn{4}{|r|}{{Continued on next page}}\\
    \hline
    \endfoot

    \bottomrule
    \endlastfoot

    Low

    &



    &

    RSK4, RSK7

    &

    \textit{RSK2}, RSK8, RSK13, RSK14\\

    \midrule

    Medium

    &

    RSK12

    &

 

    &

    RSK1, \textit{RSK3}, RSK5, RSK10, RSK11, \textbf{RSK15}, \textbf{RSK16}\\

    \midrule

    High

    &



    &

    RSK9

    &

    RSK6\\

\end{xltabular}
\label{tbl:risk_matrix}


