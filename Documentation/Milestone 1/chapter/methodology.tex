\chapter{Methodology} \label{ch:methodology}
    \section{Software Developement Methodologies}
        \subsection{Development Methodology}
            Due to the nature of our team and project, We decided to implement an agile-based development methodology.\
            As a small team, it is easy for us to communicate with each other about the project, which allows us to be
            aware of what the other members of our team are doing, both of which are key factors for choosing an agile
            methodology.\ The question There are a variety of methodologies available, each with its upsides and
            downsides, which are discussed below.\ Based on our research into these, we settled and agreed to use a
            scrum-based development methodology.\ Scrum is a combination of iterative and incremental development.\
            Allowing for a best of both worlds approach of having early builds working, but also being agile, and able
            to add requirements during the development process\cite{srivastava_2017_scrum}.

            The scrum methodology is rather simple.\ To work effectively, it requires collaboration between an appointed
            ‘Scrum Master’ and the rest of the development team, as well as a ‘Product Owner’.\ The members will work in
            close collaboration with each other, and have multiple, continuous iterations of the software builds to
            create a finished product.\ The role of the scrum master is to eliminate
            impediments\cite{srivastava_2017_scrum}, and to create the conditions that allow the development team to
            work in their most effective manner, they take a leading role in choosing what sprints the teamwork on and
            are crucial when team splitting decisions are required.

            A typical scrum workflow consists of iterative scrum cycles, typically lasting one to three weeks each.\
            Initially, a requirements backlog is created.\ This is done in collaboration with the client and is the
            documentation that describes all the requirements that the software must meet.\ Essentially, it is a
            description of the product that the team are aiming to build.\ After the initial backlog is established, the
            scrum cycles can start.\ As before, each scrum cycle lasts typically up to three weeks.\ The cycle begins
            with the team, headed by the scrum master deciding which requirements to prioritise and implement in the
            upcoming cycle.\ The aim is to have a potentially shippable/working product at the end of each sprint,
            although this may not be possible immediately due to the workload required to build a framework, following
            sprints are expected to get increasingly more successful.\ Once the team have allocated the suitable
            requirements and resources to the sprint, the sprint begins. The team works on the project, meeting each day
            in what are known as ‘scrums’ to review progress made on each day and discuss any problems or insight they
            have found on the implementation. The team will also meet and hold a sprint review, with the product owner.
            This demonstrates the product and how it has been developed and the progress since the last sprint. The team
            will also review the previous sprint, to see if any changes need to be made in anticipation of the next
            one.\cite{nuevo_2011_scrumbased}

            For our project, we will be slightly modifying the implementation of scrum. The above-mentioned workflow is
            designed as a framework for full-time software developers, which we are not. As students with other
            commitments, we will not be able to meet daily to discuss the project or work on it full-time. It would be an
            unrealistic target to hold daily scrums, but we can still follow our adaptation of the methodology. We will
            aim to hold scrums at frequent intervals, most likely every week, depending on other work needed to be completed for the university. This will allow
            us to keep track of current sprints, as well as compensate for this by being realistic in what we can
            achieve in the given sprint timeframe, and what we set as the sprint’s goals.

            Ultimately, we chose to use scrum for a variety of reasons. Primarily because we realised that as a small
            team, frequent communication would be easy to achieve, so we would be able to make use of this framework in
            the most effective manner, compared to other methodologies. We would be able to meet frequently and use that
            to ensure progress is made on the project. Scrum works well with this, which enables us to work together to
            frequently get working, testable products and allow us to implement more features that would otherwise rely
            on this. We also feel that using scrum would allow us to have a higher quality product, as the frequent
            iterations during sprints would allow us to streamline our implementation and constantly improve it, which
            also minimises risk.

            In comparison, the waterfall methodology where implementation begins much later in the lifecycle once all
            documentation and test suites are formulated. Using scrum, we start programming earlier, giving us more time
            to deal with any difficulties we encounter. It allows us to identify problems earlier, have longer to
            address them and encourage us to plan alternative implementations that avoid such development deadlocks.

            We chose scrum over other pure agile methodologies mainly because we thought it was the best fit for our
            team and situation. Other methodologies, like extreme programming, could have worked well, but didn’t suit
            our situation as students as well and added unnecessary risk to an already time-critical project. For
            example, extreme programming requires pair programming, something that would be hard for us to do
            consistently and would drastically diminish our overall productivity.

        \subsection{Hardware}
            We have chosen to base our hardware on the TinyPico implementation of the Espressif Systems ESP32 family of
            devices.\ This is mainly due to our experience with the platform, including previous university modules
            experimenting with these microcontrollers.\ To complement the TinyPico, we will need to include an array of
            input sensors, as well as ways to output information to the user.\ To detect movement, we will rely
            on accelerometer data from the pervasive ADXL family of accelerometers, namely the ADXL345 due to its
            reliable and proven performance, as well as its easy availability and team members' experience with the
            device.\ Beyond what's available on the TinyPico, we may also require additional methods of receiving and
            transiting data to and from our wearable device and the walking aid device, we have a few technologies we
            are considering, such as 433MHz Rx/Tx, or the proprietary LoRa communication system if the budget allows for
            it.

            For outputs, the wearable device will need an output LED that may be handled by the TinyPico itself, or most
            likely an external low voltage LED to allow for more flexible placement, furthermore, it will need support
            for vibration feedback, which will likely come in the format of a mobile phone style vibration motor.\ This
            will ensure the users are not startled, due to its lower power and smaller size The walking aid device will
            require to have a small speaker as an output, to prompt the user to use their walking aid.\ We can
            then either include a microphone onboard the device to record and store data to the onboard TinyPico
            storage or support the use of an external SD card to access a prerecorded message.

            One of the group members has access to a 3D Printer and has experience in 3D Design, and such we will be
            able to rapidly develop and test chassis and shells for our sensors, to more closely resemble a finished
            product whilst testing or demoing to our client.\ This is crucial due to the importance of the form factor
            and readability of the devices, as previously discussed in the document, such as minimising patient discomfort.

            The TinyPico form factor is incredibly compact and would be viable for the NONFREQ1 requirement.\ This also
            extends to the input and output sensors, which we also believe would easily fit on a wearable or walking aid
            based device.\ The low power consumption of the TinyPico and sensors will help us ensure NONFREQ2 and
            NONFREQ3 are met.\ This can be further exploited by implementing deep sleep algorithms within TinyPico's
            routine, further reducing power consumption thus extending battery life on a single charge.

        \subsection{Software}
            Whilst working on the project, our team will use a plethora of services to assist in the communication and
            management of the project.\ We have decided to use the Discord platform to keep in touch and communicate.
            This allows us to organise our channels and information streams accordingly, and even has essential tools
            for peer development such as voice/video calls, with screen sharing support.\ We started a git repository on
            GitHub, which is where our source code and work towards our project deliverables will be.\ This version
            control software allows us all to work on the tasks together or independently, and for progress to be easily
            managed, tracked and merged with everyone else's work.

            Our milestones and written deliverables will be written in Latex and stored on GitHub to allow each
            team member to use the IDE or editing environment they are most familiar with, this ranges from Overleaf, to
            Intellij, to less specific editors like ATOM or SublimeText

            The TinyPico supports Arduino C, which is a subset of C with an essentially very similar, but tailored
            feature set and command set.\ We have some experience with this in the past, however, we will keep an open mind
            towards Micropython, which promises similar levels of performance and flexibility, in an arguably easier to
            learn and use language with which we have more experience.

    \section{Testing Strategy}
        To perform successful and consistent testing across the development of our project, we have decided upon a
        testing strategy that our team will follow.\ We have designed the testing strategy around our small team,
        following the schedule detailed earlier in the document, and complies with our chosen software lifecycle
        methodology of an adapted scrum methodology.\ Our testing strategy consists of mainly integration testing, which
        we will perform after the scrum sprints.\ A sprint consists of a block of time where our
        developers will develop code for a specific feature that needs to be added to the already built system.
        Integration testing will be performed after these sprints to ensure that the features being added
        to the system do not create any additional issues within the system and that its role in the system is performed
        as expected.

        Additionally, the features being added after a sprint need to be tested throughout the sprint
        periods.\ The testing of embedded systems can be a very manual process because many of these systems
        rely upon actions from the outside world to affect how the system responds.\ We will still aim to include unit
        testing when developing our features to ensure they are working correctly before they are implemented into the
        larger system, possibly using expected real-world values in place of direct sensor data.\ Such unit tests could
        consist of checking that correct outputs are produced when certain inputs are passed into a function.\ For
        example, an input of distance walked could be passed into a function along with a boolean value that states
        whether the walking aid is moving or not.\ We would expect here that a notification is output to the walking aid
        when the distance walked is greater than 1m, but no notification is otherwise expected.\ Therefore, we could unit
        test this to ensure that the correct message is output when a certain criterion of input is passed into the
        function.\ The unit tests themselves could be classified as white box testing and will be used to test the inner
        workings of the system in an attempt to identify bugs that are hidden within the implementations of functions or
        through the use of complex code.\ However, we will also apply black-box testing to our features to test the more
        broad functionality of the component being worked on, within the scope of the system

        As previously mentioned, embedded systems involve hands-on and manual testing within a real-world scenario.
        Manual tests will allow for the effects of real-life events to be simulated, and thus the product tested during
        production.\ Such events could consist of testing the fall detection feature (which is a stretch goal but is a
        good example) which cannot simply be tested within the code itself.\ We would instead simulate the inputs of the
        device falling to the ground and then check if an emergency message has been fired by our wearable device.
        Manual tests on our embedded systems will be documented and will include the procedures taken to run the tests
        and the effects this had on the system.\ Our testing will then be evaluated to ensure that the behaviour of the
        system matches the specification and the expected outputs.
