\chapter{Testing}
\label{ch:testing}

    This chapter will detail the acceptance testing that the group carried out in order to confirm that the developed prototype adheres to the user requirements and specification that the client previously accepted in our Milestone 1 document. The chapter will be split into test cases for the different areas of the developed system. For example, a section on the test cases that specifically tested the efficiency of the systems' load times. Each test case will be assigned its own test code for ease of reference, along with a list of requirements or specifications it helps to meet, a description of the test case itself, a brief description of the acceptance criteria and the result of the test case. In the case that a test case does not completely meet the acceptance criteria, the test case will include a reason section that explains why the test did not fully meet the acceptance criteria.
    
    Despite the fact that the Milestone 1 document included an agreed user requirement and specification section, we have since come to realise that import specifications and requirements were omitted from that document. This is due to the fact that as we developed code, we gained a deeper understanding of the task at hand and further specifications and requirements became apparent. In the case that a test case aims to help solve a requirement or specification that was not included in Milestone 1, we will detail the requirement or specification without a reference to a requirement code. Otherwise the requirement will be included along with its code to allow for simple referencing to previous requirements documentation.

    Testing prior to this round of acceptance testing utilised standard embedded system testing along with integration testing. An example of this would be the testing that took place to implement the feature that plays an audio file from an SD card. A standalone system that transfers an audio file from the SD card to SPIFFS was developed initially and tested to ensure that the system itself was functional. It was at this point that we identified the issue with load times that was previously mentioned in section \ref{subsec:sdfat}. Once we had developed a full functional standalone system of the feature being implemented, we moved onto implementing that feature with the wider system. Following this, integration testing was utilised to ensure that the feature being added was functional with the rest of the system.

    The strategy for the following acceptance tests was to identify the key areas of the software system and generate test cases for those areas that would ensure that the system was able to meet the previously declared user requirements. Some key areas of the system that were being tested allowed us to identify multiple specifications and requirements that were omitted from the original requirements document. Each of the following test cases were executed multiple times to ensure robustness in the system and feature under test. We should also state that the following test cases were carried out on the final prototype. Therefore, these tests were not carried out on previously developed systems that contain standalone features, and were instead carried out on the final system that incorporates all features working simultaneously.

    \todo{Impact/Resistance testing for prototype casing}

    \section{System Boot}
    \label{sec:test_boot}

        This section will include the test cases that tested the behaviour of the system when booting, more specifically the speed of initial boot and how the system notifies the user that it is powered on.

        \vspace{1em}
        {
\bgroup
\def\arraystretch{1.5}
\centering
\begin{tabular}{| m{0.97\textwidth} |} 
    \hline
    \textbf{Test Case:} TEST\_WEARABLE\_BOOT01\\ 
    \hline
    \textbf{Assert Specification(s):} The wearable device should boot from power on without delay.\\ 
    \hline
    \textbf{Description:} This test ensures that the wearable device should power on without significant delay when the prototype is connected via USB to the developer's computer system. This simulates the device being powered on with the recharging or replacements of batteries in future.\\ 
    \hline
    \textbf{Acceptance Criteria:} 
        \begin{itemize}
            \item The wearable device should boot on power on without delay.
            \item From boot, the wearable device should instantly begin the initialisation of necessary hardware components.
        \end{itemize}\\ 
    \hline
    \textbf{Result:} \colorbox{green}{PASS}\\ 
    \hline
\end{tabular}
\egroup
}

        \vspace{4em}
        {
\bgroup
\def\arraystretch{1.5}
\centering
\begin{tabular}{| m{0.97\textwidth} |} 
    \hline
    \textbf{Test Case:} TEST\_WEARABLE\_BOOT02\\ 
    \hline
    \textbf{Assert Specification(s):} The wearable device should transition from the background to the foreground with use of the onboard LED.\\ 
    \hline
    \textbf{Description:} Partially following the tangible bits model, this test ensures that the wearable device flashes its LED on boot to represent the device entering the foreground from the background and powering on.\\ 
    \hline
    \textbf{Acceptance Criteria:} 
        \begin{itemize}
            \item On boot, the LED of the wearable device should flash green for 1 second to represent the device being powered on.
        \end{itemize}\\ 
    \hline
    \textbf{Result:} \colorbox{green}{PASS}\\ 
    \hline
\end{tabular}
\egroup
}

        \vspace{4em}
        {
\bgroup
\def\arraystretch{1.5}
\centering
\begin{tabular}{| m{0.97\textwidth} |} 
    \hline
    \textbf{Test Case:} TEST\_WALKING\_AID\_BOOT01\\ 
    \hline
    \textbf{Assert Specification(s):} The walking aid device should boot from power on without delay.\\ 
    \hline
    \textbf{Description:} This test ensures that the walking aid device should power on without significant delay when the prototype is connected via USB to the developer's computer system. This simulates the device being powered on with the recharging or replacements of batteries in future.\\ 
    \hline
    \textbf{Acceptance Criteria:} 
        \begin{itemize}
            \item The walking aid device should boot on power on without delay.
            \item From boot, the walking aid device should instantly begin the initialisation of necessary hardware components.
        \end{itemize}\\ 
    \hline
    \textbf{Result:} \colorbox{green}{PASS}\\ 
    \hline
\end{tabular}
\egroup
}

        \vspace{4em}
        {
\bgroup
\def\arraystretch{1.5}
\centering
\begin{table}[H]
\begin{tabular}{| m{0.97\textwidth} |} 
    \hline
    \textbf{Test Case:} TEST\_WALKING\_AID\_BOOT02\\ 
    \hline
    \textbf{Assert Specification(s):} The walking aid device should clearly notify surrounding users that it has been powered on.\\ 
    \hline
    \textbf{Description:} Partially following the tangible bits model, this test ensures that the walking aid device flashes its LED on boot to represent the device entering the foreground from the background and powering on.\\ 
    \hline
    \textbf{Acceptance Criteria:} 
        \begin{itemize}
            \item On boot, the LED of the walking aid device should flash green for 1 second to represent the device being powered on.
        \end{itemize}\\ 
    \hline
    \textbf{Result:} \colorbox{green}{PASS}\\ 
    \hline
\end{tabular}
\caption{Walking Aid Device Boot Test Case 2}
\end{table}
\egroup
}

    \section{System Initialisation}
    \label{sec:test_init}

        System initialisation handles the configuration of hardware components connected to each of the walking aid and wearable devices. Within this section, we will include test cases that test the functionality of the system when initialisation errors are encountered, as well as efficiency testing of the systems when performing initialisation tasks.

        \vspace{1em}
        {
\bgroup
\def\arraystretch{1.5}
\centering
\begin{tabular}{| m{0.97\textwidth} |} 
    \hline
    \textbf{Test Case:} TEST\_WEARABLE\_INIT01\\ 
    \hline
    \textbf{Assert Specification(s):} The wearable device should handle errors safely and ensure that the user can identify the issue with the device.\\ 
    \hline
    \textbf{Description:} This test ensures that the wearable device displays a debugging LED when it experiences a failure in initialising ESP-Now.\\ 
    \hline
    \textbf{Acceptance Criteria:} 
        \begin{itemize}
            \item The wearable device should display a solid red LED light when it cannot initialise ESP-Now successfully.
        \end{itemize}\\ 
    \hline
    \textbf{Result:} \colorbox{green}{PASS}\\ 
    \hline
\end{tabular}
\egroup
}

        \vspace{4em}
        {
\bgroup
\def\arraystretch{1.5}
\centering
\begin{tabular}{| m{0.97\textwidth} |} 
    \hline
    \textbf{Test Case:} TEST\_WEARABLE\_INIT02\\ 
    \hline
    \textbf{Assert Specification(s):} The wearable device should handle errors safely and ensure that the user can identify the issue with the device.\\ 
    \hline
    \textbf{Description:} This test ensures that the wearable device displays a debugging LED when it experiences an issue in registering the walking aid device as a peer for communication.\\ 
    \hline
    \textbf{Acceptance Criteria:} 
        \begin{itemize}
            \item The wearable device should display a solid orange LED light when the wearable device cannot add the walking aid device as a peer.
        \end{itemize}\\ 
    \hline
    \textbf{Result:} \colorbox{green}{PASS}\\ 
    \hline
\end{tabular}
\egroup
}

        \vspace{4em}
        {
\bgroup
\def\arraystretch{1.5}
\centering
\begin{table}[H]
\begin{tabular}{| m{0.97\textwidth} |} 
    \hline
    \textbf{Test Case:} TEST\_WEARABLE\_INIT03\\ 
    \hline
    \textbf{Assert Specification(s):} The wearable device should not take a considerable amount of time to boot and initialise.\\ 
    \hline
    \textbf{Description:} This test ensures that the wearable device does not take a considerable amount of time in initalising the ESP-Now communication protocol.\\ 
    \hline
    \textbf{Acceptance Criteria:} 
        \begin{itemize}
            \item The wearable device should take less than 5 seconds to inititialise the ESP-Now communication protocol.
        \end{itemize}\\ 
    \hline
    \textbf{Result:} \colorbox{green}{PASS}\\ 
    \hline
\end{tabular}
\caption{Wearable Device Initialisation Test Case 3}
\end{table}
\egroup
}

        \vspace{4em}
        {
\bgroup
\def\arraystretch{1.5}
\centering
\begin{tabular}{| m{0.97\textwidth} |} 
    \hline
    \textbf{Test Case:} TEST\_WALKING\_AID\_INIT01\\ 
    \hline
    \textbf{Assert Specification(s):} Should the walking aid device experience a failure when initialising the ESP-Now communications protocol, a debugging LED light should be displayed to help the user understand the issue.\\ 
    \hline
    \textbf{Description:} This test ensures that the walking aid device displays a debugging LED when it experiences a failure in initialising ESP-Now.\\ 
    \hline
    \textbf{Acceptance Criteria:} 
        \begin{itemize}
            \item The walking aid device should display a solid red LED light when it cannot initialise ESP-Now successfully.
        \end{itemize}\\ 
    \hline
    \textbf{Result:} \colorbox{green}{PASS}\\ 
    \hline
\end{tabular}
\egroup
}

        \vspace{4em}
        {
\bgroup
\def\arraystretch{1.5}
\centering
\begin{table}[H]
\begin{tabular}{| m{0.97\textwidth} |} 
    \hline
    \textbf{Test Case:} TEST\_WALKING\_AID\_INIT02\\ 
    \hline
    \textbf{Assert Specification(s):} Should the walking aid device experience an issue when attempting to add the wearable device as an ESP-Now peer, a debugging LED light should be displayed to help the user understand the issue.\\ 
    \hline
    \textbf{Description:} This test ensures that the walking aid device displays a debugging LED when it experiences an issue in registering the wearable device as a peer for communication.\\ 
    \hline
    \textbf{Acceptance Criteria:} 
        \begin{itemize}
            \item The walking aid device should display a solid orange LED light when the wearable device cannot add the wearable device as a peer.
        \end{itemize}\\ 
    \hline
    \textbf{Result:} \colorbox{green}{PASS}\\ 
    \hline
\end{tabular}
\caption{Walking Aid Device Initialisation Test Case 2}
\end{table}
\egroup
}

        \vspace{4em}
        {
\bgroup
\def\arraystretch{1.5}
\centering
\begin{tabular}{| m{0.97\textwidth} |} 
    \hline
    \textbf{Test Case:} TEST\_WEARABLE\_INIT03\\ 
    \hline
    \textbf{Assert Specification(s):} Should the walking aid device experience a failure when attempting to initialise the SD card, the system should stall and notify the user of the issue.\\ 
    \hline
    \textbf{Description:} This test ensures that when the walking aid device experiences a failure in initialising the SD card, the system blocks and an LED is displayed to allow for debugging.\\ 
    \hline
    \textbf{Acceptance Criteria:} 
        \begin{itemize}
            \item On SD card initialisation failure, the system should block and display a solid blue LED light on the audio shield.
        \end{itemize}\\ 
    \hline
    \textbf{Result:} \colorbox{green}{PASS}\\ 
    \hline
\end{tabular}
\egroup
}

        \vspace{4em}
        {
\bgroup
\def\arraystretch{1.5}
\centering
\begin{tabular}{| m{0.97\textwidth} |} 
    \hline
    \textbf{Test Case:} TEST\_WALKING\_AID\_INIT04\\ 
    \hline
    \textbf{Assert Specification(s):} The walking aid device should not take a considerable amount of time to boot and initialise.\\ 
    \hline
    \textbf{Description:} This test ensures that the walking aid device does not take a considerable amount of time in initalising communication to the SD card unless an exception has been thrown.\\ 
    \hline
    \textbf{Acceptance Criteria:} 
        \begin{itemize}
            \item The walking aid device should take less than 5 seconds to inititialise communication with the SD card should there be no errors.
        \end{itemize}\\ 
    \hline
    \textbf{Result:} \colorbox{green}{PASS}\\ 
    \hline
\end{tabular}
\egroup
}

        \vspace{4em}
        {
\bgroup
\def\arraystretch{1.5}
\centering
\begin{tabular}{| m{0.97\textwidth} |} 
    \hline
    \textbf{Test Case:} TEST\_WALKING\_AID\_INIT05\\ 
    \hline
    \textbf{Assert Specification(s):} The walking aid device should not take a considerable amount of time to boot and initialise.\\ 
    \hline
    \textbf{Description:} This test ensures that the walking aid device does not take a considerable amount of time in initalising the ESP-Now communication protocol.\\ 
    \hline
    \textbf{Acceptance Criteria:} 
        \begin{itemize}
            \item The walking aid device should take less than 5 seconds to inititialise the ESP-Now communication protocol.
        \end{itemize}\\ 
    \hline
    \textbf{Result:} \colorbox{green}{PASS}\\ 
    \hline
\end{tabular}
\egroup
}

        \vspace{4em}
        {
\bgroup
\def\arraystretch{1.5}
\centering
\begin{tabular}{| m{0.97\textwidth} |} 
    \hline
    \textbf{Test Case:} TEST\_WALKING\_AID\_INIT06\\ 
    \hline
    \textbf{Assert Specification(s):} The walking aid device should not take a considerable amount of time to boot and initialise.\\ 
    \hline
    \textbf{Description:} This test ensures that the walking aid device does not take a considerable amount of time to transfer the audio file from the SD card to the SPIFFS storage.\\ 
    \hline
    \textbf{Acceptance Criteria:} 
        \begin{itemize}
            \item The walking aid device should take less than 30 seconds to transfer the audio file from the SD card to SPIFFS storage.
        \end{itemize}\\ 
    \hline
    \textbf{Result:} \colorbox{green}{PASS}\\ 
    \hline
\end{tabular}
\egroup
}

    \section{Walking Detection}
    \label{sec:test_walking_detection}

        With the walking detection feature of the system being the most crucial to implement correctly, this section will include acceptance tests for the system's walking detection functionality. The wearable device will be tested for its walking detection, with the walking aid device being tested for its tap detecting feature.

        \vspace{1em}
        {
\bgroup
\def\arraystretch{1.5}
\centering
\begin{table}[H]
\begin{tabular}{| m{0.97\textwidth} |} 
    \hline
    \textbf{Test Case:} TEST\_WEARABLE\_WALK01\\ 
    \hline
    \textbf{Assert Specification(s):} FREQ1: The wearable device should detect when a patient has walked more than 1 metre before communicating with the walking aid.\\ 
    \hline
    \textbf{Description:} This test ensures that the wearable device is able to detect changes in gravity as steps through the accelerometer.\\ 
    \hline
    \textbf{Acceptance Criteria:} 
        \begin{itemize}
            \item The wearable device should detect steps when the user is walking with the device attached to them.
        \end{itemize}\\ 
    \hline
    \textbf{Result:} \colorbox{Dandelion}{PARTIAL PASS}\\ 
    \hline
    \textbf{Reason:} Despite the accelerometer being able to detect steps when the user is moving with the device attached to them, the system inherits similar issues to step counter systems employed in some of the most well known mobile devices on the market. That is, the user can be still, but the frequent motion of the accelerometer can cause steps to be counted.\\
    \hline
\end{tabular}
\caption{Wearable Device Walking Detection Test Case 1}
\end{table}
\egroup
}

        \vspace{4em}
        {
\bgroup
\def\arraystretch{1.5}
\centering
\begin{tabular}{| m{0.97\textwidth} |} 
    \hline
    \textbf{Test Case:} TEST\_WEARABLE\_WALK02\\ 
    \hline
    \textbf{Assert Specification(s):} FREQ1: The wearable device should detect when a patient has walked more than 1 metre before communicating with the walking aid.\\ 
    \hline
    \textbf{Description:} This test ensures that the wearable device detects when the user has walked more than 1 metre before communicating with the walking aid device.\\ 
    \hline
    \textbf{Acceptance Criteria:} 
        \begin{itemize}
            \item The wearable device should send a message to the walking aid device when the user has walked more than 5 steps in a 10 second period.
        \end{itemize}\\ 
    \hline
    \textbf{Result:} \colorbox{Dandelion}{PARTIAL PASS}\\ 
    \hline
    \textbf{Reason:} In normal walking conditions, the wearable device sends a message to the walking aid device after walking over 1 metre. However, the same issue flagged in TEST_WEARABLE_01 can be identified here. That is, should the user move the wearable device with enough force, steps will be detected even without them walking. This issue can be limited with adjustments to thresholds and through wearing of the device on the ankle.\\
    \hline
\end{tabular}
\egroup
}

        \vspace{4em}
        {
\bgroup
\def\arraystretch{1.5}
\centering
\begin{tabular}{| m{0.97\textwidth} |} 
    \hline
    \textbf{Test Case:} TEST\_WALKING\_AID\_WALK01\\ 
    \hline
    \textbf{Assert Specification(s):} A reminder should be sent to the user when they start moving without their walking aid.\\ 
    \hline
    \textbf{Description:} This test ensures that the walking aid device is able to detect that it is moving due to single tap detection in the accelerometer.\\ 
    \hline
    \textbf{Acceptance Criteria:} 
        \begin{itemize}
            \item The walking aid device should detect when it is being lifted and placed on the ground by the user.
        \end{itemize}\\ 
    \hline
    \textbf{Result:} \colorbox{green}{PASS}\\ 
    \hline
\end{tabular}
\egroup
}

        \vspace{4em}
        {
\bgroup
\def\arraystretch{1.5}
\centering
\begin{tabular}{| m{0.97\textwidth} |} 
    \hline
    \textbf{Test Case:} TEST\_WALKING\_AID\_WALK02\\ 
    \hline
    \textbf{Assert Specification(s):} The system should detect if the walking aid is moving too before playing an alarm.\\ 
    \hline
    \textbf{Description:} This test ensures that the reminder cannot be played if movement was detected 10 seconds prior to a message being received from the wearable device.\\ 
    \hline
    \textbf{Acceptance Criteria:} 
        \begin{itemize}
            \item Movement detected within 10 seconds prior to a message being sent by the wearable device should stop the reminder from being played.
        \end{itemize}\\ 
    \hline
    \textbf{Result:} \colorbox{green}{PASS}\\ 
    \hline
\end{tabular}
\egroup
}

        \vspace{4em}
        {
\bgroup
\def\arraystretch{1.5}
\centering
\begin{tabular}{| m{0.97\textwidth} |} 
    \hline
    \textbf{Test Case:} TEST\_WALKING\_AID\_WALK03\\ 
    \hline
    \textbf{Assert Specification(s):} The system should allow the user enough time to get to their walking aid device before a reminder is played.\\ 
    \hline
    \textbf{Description:} This test ensures that the user is offered time to use the walking aid after their walking has been detected, before playing the reminder.\\ 
    \hline
    \textbf{Acceptance Criteria:} 
        \begin{itemize}
            \item Movement detected within 10 seconds after a message has been sent by the wearable device should stop the reminder from being played.
        \end{itemize}\\ 
    \hline
    \textbf{Result:} \colorbox{green}{PASS}\\ 
    \hline
\end{tabular}
\egroup
}

    \section{Communications}
    \label{sec:test_comms}

        To allow the walking aid device to understand when the wearable device is moving, a communication system needed to be developed. To ensure that the communication system is functioning as intended, we designed test cases that ensure that both devices are able to communicate with each other when necessary. Those test cases can be seen below.

        \vspace{1em}
        {
\bgroup
\def\arraystretch{1.5}
\centering
\begin{tabular}{| m{0.97\textwidth} |} 
    \hline
    \textbf{Test Case:} TEST\_WEARABLE\_COMMS01\\ 
    \hline
    \textbf{Assert Specification(s):} 
    
    \begin{itemize}
        \item FREQ1: The wearable device should detect when a patient has walked more than 1 metre before communicating with the walking aid.
        \item FREQ5: The wearable device should communicate to the
        walking aid device to let it know when it's started
        moving.
    \end{itemize}\\ 
    \hline
    \textbf{Description:} This test ensures that the wearable device attempts to send a message to the walking aid device should more than 5 steps be detected in a 10 second period.\\ 
    \hline
    \textbf{Acceptance Criteria:} 
        \begin{itemize}
            \item The wearable device should broadcast a message to the walking aid device if it detects more than 5 steps in a 10 second period.
        \end{itemize}\\ 
    \hline
    \textbf{Result:} \colorbox{green}{PASS}\\ 
    \hline
\end{tabular}
\egroup
}

        \vspace{4em}
        {
\bgroup
\def\arraystretch{1.5}
\centering
\begin{table}[H]
\begin{tabular}{| m{0.97\textwidth} |} 
    \hline
    \textbf{Test Case:} TEST\_WEARABLE\_COMMS02\\ 
    \hline
    \textbf{Assert Specification(s):} FREQ3: The wearable device should include a solution for deaf people that still reminds them to take their walking aid with them without the need for an audio alarm.\\ 
    \hline
    \textbf{Description:} This test ensures that the wearable device can accept messages from the walking aid device.\\ 
    \hline
    \textbf{Acceptance Criteria:} 
        \begin{itemize}
            \item The wearable device should broadcast a message to the walking aid device if it detects more than 5 steps in a 10 second period.
        \end{itemize}\\ 
    \hline
    \textbf{Result:} \colorbox{green}{PASS}\\ 
    \hline
\end{tabular}
\caption{Wearable Device Communications Test Case 2}
\end{table}
\egroup
}

        \vspace{4em}
        {
\bgroup
\def\arraystretch{1.5}
\centering
\begin{tabular}{| m{0.97\textwidth} |} 
    \hline
    \textbf{Test Case:} TEST\_WALKING\_AID\_COMMS01\\ 
    \hline
    \textbf{Assert Specification(s):} FREQ3: The wearable device should include a solution for deaf people that still reminds them to take their walking aid with them without the need for an audio alarm.\\ 
    \hline
    \textbf{Description:} This test ensures that the walking aid device attempts to send a message to the wearable device to vibrate should a reminder need to be played but the user is hard of hearing.\\ 
    \hline
    \textbf{Acceptance Criteria:} 
        \begin{itemize}
            \item The walking aid device should broadcast a message to the wearable device if the wearable should vibrate.
        \end{itemize}\\ 
    \hline
    \textbf{Result:} \colorbox{green}{PASS}\\ 
    \hline
\end{tabular}
\egroup
}

        \vspace{4em}
        {
\bgroup
\def\arraystretch{1.5}
\centering
\begin{tabular}{| m{0.97\textwidth} |} 
    \hline
    \textbf{Test Case:} TEST\_WALKING\_AID\_COMMS02\\ 
    \hline
    \textbf{Assert Specification(s):} 
    
    \begin{itemize}
        \item FREQ1: The wearable device should detect when a patient has walked more than 1 metre before communicating with the walking aid.
        \item FREQ5: The wearable device should communicate to the
        walking aid device to let it know when it's started
        moving.
    \end{itemize}\\ 
    \hline
    \textbf{Description:} This test ensures that the walking aid device can accept messages from the wearable device.\\ 
    \hline
    \textbf{Acceptance Criteria:} 
        \begin{itemize}
            \item The walking aid device should detect the reminder message sent by the wearable device.
        \end{itemize}\\ 
    \hline
    \textbf{Result:} \colorbox{green}{PASS}\\ 
    \hline
\end{tabular}
\egroup
}

    \section{Reminders}
    \label{sec:test_reminders}

        This section will include acceptance tests for the system's reminder feature. The basis of this project is to ensure that patients are reminded to take their walking aids with them when they start walking. Therefore, it is imperative that this feature is complete. The wearable device includes a vibration reminder should the user be hard of hearing, and the walking aid device includes an audio reminder system.

        \vspace{1em}
        {
\bgroup
\def\arraystretch{1.5}
\centering
\begin{table}[H]
\begin{tabular}{| m{0.97\textwidth} |} 
    \hline
    \textbf{Test Case:} TEST\_WEARABLE\_REM01\\ 
    \hline
    \textbf{Assert Specification(s):} FREQ3: The wearable device should include a solution for deaf people that still reminds them to take their walking aid with them without the need for an audio alarm.\\ 
    \hline
    \textbf{Description:} This test ensures that the wearable device vibrates when prompted to by the walking aid device.\\ 
    \hline
    \textbf{Acceptance Criteria:} 
        \begin{itemize}
            \item The wearable device should vibrate when a message has been received from the walking aid device.
        \end{itemize}\\ 
    \hline
    \textbf{Result:} \colorbox{green}{PASS}\\ 
    \hline
\end{tabular}
\caption{Wearable Device Reminder Test Case 1}
\end{table}
\egroup
}

        \vspace{4em}
        {
\bgroup
\def\arraystretch{1.5}
\centering
\begin{tabular}{| m{0.97\textwidth} |} 
    \hline
    \textbf{Test Case:} TEST\_WALKING\_AID\_REM01\\ 
    \hline
    \textbf{Assert Specification(s):} FREQ2: Patients should be alerted with the voice of a friend, carer or relative to avoid startling them.\\ 
    \hline
    \textbf{Description:} This test ensures that the walking aid device plays the audio file when a message has been received from the wearable device and the walking aid has not been moved.\\ 
    \hline
    \textbf{Acceptance Criteria:} 
        \begin{itemize}
            \item The walking aid device should play the audio file when a message is received from the wearable device and the walking aid has not met the criteria for being moved.
        \end{itemize}\\ 
    \hline
    \textbf{Result:} \colorbox{green}{PASS}\\ 
    \hline
\end{tabular}
\egroup
}

    \section{Impact and Strength}
    \label{sec:test_strength}

        This section will include acceptance tests for the system's physical attributes, due to the nature of the device, it must be able to handle falls and impacts, and continue to work afterwards.

        \vspace{1em}
        {
\bgroup
\def\arraystretch{1.5}
\centering
\begin{table}[H]
\begin{tabular}{| m{0.97\textwidth} |} 
    \hline
    \textbf{Test Case:} TEST\_WEARABLE\_IMPCT01\\ 
    \hline
    \textbf{Assert Specification(s):} The system should be able to be fallen onto from three feet (0.92m) whilst on hard flooring without suffering critical damage. The weight falling should be 100kg\\ 
    \hline
    \textbf{Description:} This test ensures that the device would survive a patient falling whilst the device is on their person.\\ 
    \hline
    \textbf{Acceptance Criteria:} 
        \begin{itemize}
            \item Device continues to work after the fall
            \item No excessive damage to enclosure or mount
        \end{itemize}\\ 
    \hline
    \textbf{Result:} \colorbox{green}{PASS}\\ 
    \hline
\end{tabular}
\caption{Wearable Device Impact Test Case 1}
\end{table}
\egroup
}

        \vspace{4em}
        {
\bgroup
\def\arraystretch{1.5}
\centering
\begin{table}[H]
\begin{tabular}{| m{0.97\textwidth} |} 
    \hline
    \textbf{Test Case:} TEST\_WALKING\_AID\_IMPCT01\\ 
    \hline
    \textbf{Assert Specification(s):} The system should be able to be dropped from three feet (0.92m) onto hard flooring without suffering critical damage.\\ 
    \hline
    \textbf{Description:} This test ensures that the device would survive the walking aid being dropped, such as the walking stick its attached to being released.\\ 
    \hline
    \textbf{Acceptance Criteria:} 
        \begin{itemize}
            \item Device continues to work after the fall
            \item No excessive damage to enclosure or mount
        \end{itemize}\\ 
    \hline
    \textbf{Result:} \colorbox{green}{PASS}\\ 
    \hline
\end{tabular}
\caption{Walking Aid Device Impact Test Case 1}
\end{table}
\egroup
}

