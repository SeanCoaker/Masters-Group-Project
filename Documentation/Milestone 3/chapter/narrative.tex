\chapter{Narrative and Reflective Account}
\label{ch:narrative}

    In this chapter, we will provide a detailed narrative and reflective account as well as a perspective on the entire process of developing the Walking Aid Usage Prompt System. Specifically, in this chapter, we will describe the development process, from the beginning to the end, along with all the problems we encountered, and how we solved them.

    \section{Narrative Account}
    \label{sec:narrative_account}

        The following section will detail a narrative account of the process of developing this project. We will begin by evaluating the final progress of the project against the requirements we specified in our Milestone 1 document \cite{coaker}, before moving onto discuss the problems we faced during project development and how we attempted to overcome those problems. Finally, we will provide a conclusion of how the project went and what we learned from it.

        \subsection{Evaluating the Project}
        \label{subsec:evaluation}

            This section will evaluate whether the project was a success or not by listing our requirements from our Milestone 1 document \cite{coaker}, and comparing them to the final progress of the project. We will detail whether those requirements have been met and why, before concluding the section with a body of text that evaluates our overall satisfaction with the outcome of the project. The listed requirements will also include some requirements that were omitted from the Milestone 1 document \cite{coaker} and were identified later into the project's lifecycle. The list of requirements, progress made to fulfill those requirements, and the details behind how that progress was made is listed below.

            \vspace{2em}
            \bgroup
            \def\arraystretch{1.5}
            \begin{tabular}{| p{0.7\linewidth} | c |} 
                \hline

				\textbf{FREQ1: The wearable device should detect when a patient has walked more than 1 metre before communicating with the walking aid.} & \colorbox{green}{100\% Complete.}\\ 

                \hline

                \multicolumn{2}{| p{0.9\linewidth} |}{We developed the wearable device to include an ADXL345 accelerometer that would provide the basis for detecting when the patient is walking. Utilising the single-tap detection feature of the accelerometer, we identify changes in gravity above our set threshold as a step. When each step is detected, it is added to a step counter which is used to identify if a user has walked more than 5 metres within a 10 second period. The check to identify if the user has walked 5 steps in a 10 second period ensures that we safely know that the user has walked more than 1 metre. Only when this has occurred does the system send a message to the walking aid device.
                
                However, there are some issues with this system as described previously in this document. Following the inherent issues of step counters within smartphone devices, the shaking of our wearable device can cause steps to be detected even without walking. However, this issue is very difficult to avoid without the large budgets needed to provide more research and development within this area.}\\

                \hline
				 
			\end{tabular}
            \egroup

            \vspace{2em}
            \bgroup
            \def\arraystretch{1.5}
            \begin{tabular}{| p{0.7\linewidth} | c |} 
                \hline

				\textbf{FREQ2: Patients should be alerted with the voice of a friend, carer or relative to avoid startling them.} & \colorbox{green}{100\% Complete.}\\ 

                \hline

                \multicolumn{2}{| p{0.9\linewidth} |}{The success of this requirement is based heavily on what message carers, relatives and friends decide to use as reminders for the patient. However, with the inclusion of our SD card reading and audio playing system, we feel we have provided the necessary functionality to allow voice recordings to be played as reminders from our walking aid device. Therefore, we class this requirement as completed.}\\

                \hline
				 
			\end{tabular}
            \egroup

            \vspace{2em}
            \bgroup
            \def\arraystretch{1.5}
            \begin{tabular}{| p{0.7\linewidth} | c |} 
                \hline

				\textbf{FREQ3: The wearable device should include a solution for deaf people that still reminds them to take their walking aid with them without the need for an audio alarm.} & \colorbox{green}{100\% Complete.}\\ 

                \hline

                \multicolumn{2}{| p{0.9\linewidth} |}{We feel that this requirement has been completed due to the fact that we have implemented two-way communication between our walking aid and wearable devices, that allows for the vibration motor on the wearable device to vibrate, replacing the audio reminder for deaf patients. When the walking aid device receives a message from the wearable declaring that the user has started walking, and the walking aid device does not detect any movement itself, then rather than playing an audio reminder it sends a message back to the wearable asking it to vibrate.}\\

                \hline
				 
			\end{tabular}
            \egroup

    \section{Requirements and Topic}
    \label{sec:reqs}

    The first thing we did was to arrange a meeting with the client, who in this case was a representative of Bangor Health Clinic. The reason for this meeting is to understand better the topic and also gather the requirements from the client. At first, we thought of two possible outcomes, according to the client's requirements, developing a wearable device or a non-wearable device. The first one is a device similar to a watch that will include an accelerometer, which will be responsible for calculating the user's movement, and the second one is a pressure pad, that will detect when the user is sitting or not to it. We proposed to Bangor Health Clinic, the first solution rather than the second, as the second might need extra information to calculate the user's movement, for instance, if the user stands up there is no way of understanding if the user is walking, or creating a solution in which the pad can be easily cleaned up. All of these reasons were also very difficult for us to implement, so we chose the first solution as it is the more passable solution when it comes to our skills.

    \section{Development}
    \label{sec:development}

    The next step is the development of the proposed solution, including the methodology we followed. The development of the Walking Aid System was tough for our team, to complete in time, as we lost valuable time at the beginning of the project while waiting for the proper equipment to arrive. Fortunately, we moved the project forward in both software and hardware development, developing prototypes, after their arrival. The loss of that time was something from which our timetable could never recover. Although we realized that working on a project that includes both software and hardware development may necessitate additional time, the delivery and arrival of the equipment may take extra time; therefore, the project plan must be adaptable to those dates. The development is split into three sections, the first section contains all the problems we encountered for the software development, the second section contains all the problems we encountered for the hardware development, and the final one contains all the problems we encountered for the risk analysis. Each section will highlight how we solved the problem if it was solved, and what we learnt from that solution.

    \subsection{Software Development}
    \label{ssec:software}

    Working on this type of project that requires software and hardware development was very challenging for us, especially when it comes to hardware selection, which was the first problem we faced. Because the initial budget we required exceeded the one allocated by the customer, we devised a method to reduce the ordering budget by borrowing two TinyPICOs from the University. Having now all the equimpent we needed to find a solution on how these two systems will communicate with each other. Choosing a communication protocol was a key challenge since we required one that would not exceed our budget while still being extremely precise at a long-range. As a result, we chose the ESP-Now protocol for communication between our TinyPICOs since it is simple to integrate with ESP32 chips and has a range of 220 meters.

    % Add more information about walking detection system, problems and solutions.

    \subsection{Hardware Development}
    \label{ssec:hardware}

    % What problems we encountered during this phase, and what are our solutions.
    % What we learnt from the problem(s) and the solution(s)

    \subsection{Risk Analysis}
    \label{ssec:}

    % What problems we encountered while defining risks
    When we defined our risks in Milestone 1 \cite{coaker}, we ran into issues with our budget and hardware procurement, which caused us to lose development time as mentioned in the previous section \ref{sec:development}. However, in Milestone 2 \cite{mile2}, we revised our risks because the loss of development time was still affecting our schedule, so we needed to conduct team meetings so we could oversee the project's progress, whether we were still behind schedule or not.

    \section{Reflection}
    \label{sec:Reflection}

    From this group project, the first thing we learnt from developing a project for a client is that working in groups is very difficult, especially when not all members contribute at the same level. This is something that happened in this project, as some members did not pull their weight, resulting in many incomplete tasks being left to be completed at the last minute. This might have happened since everyone's schedules were different, but it doesn't imply it's not a reasonable reason for everyone's workload. Of course, if all of the team members were equally devoted to this project, we would have greater outcomes and not have as much work to do near the finish.

    Needless to say, given this was a group project, we had to hold meetings to maintain track of the project's development. Using Discord for our meetings eased and simplified things for us. In addition, we used GitHub, which improved our git expertise and made it easy for us to divide the documentation work and all of the project's work in general.

    Although, this project gave us the chance to implement software engineering principles we learnt through the years of our studies, such as software design, choosing a proper methodology, developing a project plan and also working as a team.
