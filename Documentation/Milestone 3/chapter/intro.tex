\chapter{Introduction}

The opportunity of developing the walking aid usage prompt system emerged when Bangor Health Clinic reached out to us in need of a solution that helps remind dementia patients to use their walking aids when walking. Since that initial meeting, we developed a bi-device system that detects when the patient is walking without their walking aid, and plays an audio reminder to the patient, or a vibration reminder through a wearable device if the patient is deaf or uncomfortable with noise. 

The bi-device system encompasses a device to be attached to the patient's walking aid and another device to be attached to the body or clothes of the user. Utilising the changes in gravity through accelerometers, we have created a system that is able to detect when the patient has started walking, and whether the walking aid has also moved. We implemented an algorithm that signifies that the patient is walking when the accelerometer detects that they have made five steps in the space of a 10 second period. Communication is then made to the walking aid device, which runs a check to see if it has been moved in the last 10 seconds, or is moved in the next 10 seconds. If neither occur, an audio reminder is played to the user, or the wearable device receives a message back from the walking aid device asking it to vibrate for patients who are hard of hearing. 

The following document consists of a design section, where we detail the design decisions we made for each device, an acceptance testing section, which will contain evidence that the system was tested to ensure full functionality, and a narrative and reflective account section, where we provide a narrative account of the process of developing the project and a reflection on what we could have done better had we avoided mistakes and been offered more resources to complete the project.